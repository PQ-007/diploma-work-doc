\addtotoc{Хураангуй}

\vspace{10mm}

\begin{center}
\textbf{\large Хураангуй}
\end{center}

\sloppy
Энэхүү дипломын ажлын зорилго нь КУТ-ийн оюутнуудад зориулсан “FutureHub” нэртэй суралцах үйл явц, төсөл хэрэгжилт, мэдлэгийн удирдлагыг дэмжих цогц веб болон мобайл платформыг зохион бүтээж, хөгжүүлэхэд оршино. Оюутнууд мэргэжлийн хичээл судлах, япон хэл дээрх сурах бичиг ашиглах, төсөл гүйцэтгэх, судалгааны ажил хийх явцдаа мэргэжлийн нэр томьёоны ойлголт дутмаг байх, чанартай орчуулгатай агуулга хомсдох, мэдээлэл тархай бутархай байх, хийсэн ажлаа системтэйгээр баримтжуулах нэгдсэн орчин дутмаг зэрэг хүндрэлүүдтэй тулгардаг.

Эдгээр асуудлыг шийдвэрлэх зорилгоор уг системийг флашкарт мобайл аппликейшн болон веб платформ гэсэн хоёр үндсэн хэсгээс бүрдүүлэн загварчилсан. Флашкарт аппликейшн нь зайтай давталтын алгоритм (Spaced Repetition System)-д суурилан үг хэллэг, мэргэжлийн нэр томьёог урт хугацаанд тогтоох боломжийг бүрдүүлнэ. Мөн OCR технологиор зургаас текст таних, мэргэжлийн толь бичиг ашиглах, text-to-speech технологид суурилсан аудио давталт хийх зэрэг нэмэлт боломжуудыг агуулна.

Веб платформ нь блог, нийтлэл, хамтын толь бичиг, төслийн хөгжүүлэлтийн тэмдэглэл (project log), хэлэлцүүлэг, тэмцээн уралдаан, хэрэглэгчийн хувь нэмрийг үнэлэх систем (contribution ranking), профайл-портфолио зэрэг модуль хэсгүүдээс бүрдэнэ. Ингэснээр оюутнууд мэдлэгээ төвлөрүүлэн хадгалах, хамтран суралцах, өөрийн хөгжлийг баримтжуулах, ур чадвараа нотлох дижитал орчин бүрдэнэ.

Системийг хөгжүүлэхдээ архитектурын загварчлал, өгөгдлийн сангийн зохион байгуулалт, прототип хөгжүүлэлт, туршилт үнэлгээний аргачлалыг ашигласан. Судалгааны үр дүнд оюутнуудын суралцах үйл явцыг илүү бүтэцтэй, үр ашигтай болгох боломжтой дижитал платформын суурь шийдэл боловсруулагдсан болно.

\vspace{3mm}
\textit{Түлхүүр үгс:} Платформ, Веб систем, Мобайл аппликейшн, Флашкарт, Толь бичиг, Зайтай давталт, Боловсролын технологи

\vspace{10mm}

\begin{center}
\textbf{\large Abstract}
\end{center}

\begin{english}
This thesis aims to design and implement an integrated web and mobile platform called “FutureHub” to support students of KUT in learning, project development, and knowledge management. Students often face difficulties such as limited understanding of technical terminology in Japanese textbooks, lack of high-quality Mongolian learning resources, fragmented information sources, and the absence of a unified environment to systematically document and demonstrate their work.

To address these challenges, the proposed system consists of two main components: a flashcard-based mobile application and a web platform. The mobile application utilizes a Spaced Repetition System (SRS) to enhance long-term retention of vocabulary and technical terms. Additional features include OCR-based text extraction, a collaborative technical dictionary, and text-to-speech functionality for audio-based review.

The web platform integrates blogs, articles, a collaborative dictionary, project logs, discussions, contests, a contribution ranking system, and a profile-portfolio module. This environment enables students to centralize knowledge, collaborate effectively, document their development process, and build a structured academic portfolio.

The system was developed using architectural design principles, database modeling, prototype implementation, and user-based evaluation. As a result, a foundational digital solution was created to support structured and efficient learning among students.

\vspace{3mm}
\textit{Keywords:} Platform, Web System, Mobile Application, Flashcard, Dictionary, Spaced Repetition, Educational Technology
\end{english}