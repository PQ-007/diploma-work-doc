% ===========================================================
%  ХУРААНГУЙ (Summary — 150–250 үг)
% ===========================================================
\addtotoc{Хураангуй}

\vspace{10mm}

\begin{center}
\textbf{\large Хураангуй}
\end{center}

\sloppy
Энэхүү дипломын ажлын зорилго нь КУТ-ийн оюутнуудад зориулсан "FutureHub" нэртэй бие даан суралцах, төсөл судалгаа туршилт хийх, тэдгээрийн тэмдэглэл хөтлөх, тэмцээн уралдаанд оролцох зэргээр мэргэжлийн ур чадвар, мэдлэгээ хөгжүүлэхэд дэмжлэг үзүүлэх агуулга бүхий цогц платформ хөгжүүлэхэд оршино. 

Оюутнууд мэргэжлийн хичээл судлах, япон хэл дээрх сурах бичиг ашиглах, төсөл гүйцэтгэх, судалгааны ажил хийх явцдаа мэргэжлийн нэр томьёоны ойлголт дутмаг байх, чанартай орчуулгатай агуулга хомсдох, мэдээлэл хангалтгүй, тодорхой бус байх, хийсэн ажлаа системтэйгээр баримтжуулах нэгдсэн орчин дутмаг зэрэг хүндрэлүүдтэй тулгардаг.

Эдгээр асуудлыг шийдвэрлэх зорилгоор уг системийг флашкарт мобайл аппликейшн болон веб платформ гэсэн хоёр үндсэн хэсгээс бүрдүүлэн загварчилсан. Флашкарт аппликейшн нь Spaced Repetition System-д суурилан үг хэллэг, мэргэжлийн нэр томьёо ба түүний агуулгыг урт хугацаанд тогтоох боломжийг бүрдүүлнэ. Мөн веб платформ дээр нийтлэгдэж буй нийтлэлийг унших, OCR технологиор зургаас текст таних, мэргэжлийн толь бичиг ашиглах, text-to-speech технологид суурилсан аудиогоор флашкартуудыг уншин хадгалан сонсон давтах зэрэг нэмэлт боломжуудыг агуулна.

Веб платформ нь блог, нийтлэл, хамтын толь бичиг, төслийн хөгжүүлэлтийн тэмдэглэл (project log), хэлэлцүүлэг, тэмцээн уралдаан, хэрэглэгчдийг платфорд оруулж буй контентуудад тулгуурлан үнэлэх систем (contribution ranking), профайл-портфолио зэрэг модуль хэсгүүдээс бүрдэнэ. Ингэснээр оюутнууд мэдлэгээ төвлөрүүлэн хадгалах, хамтран суралцах, өөрийн хөгжлийг баримтжуулах, ур чадвараа нотлох дижитал орчин бүрдэнэ.

Системийг хөгжүүлэхдээ архитектурын загварчлал, өгөгдлийн сангийн зохион байгуулалт, прототип хөгжүүлэлт, туршилт үнэлгээний аргачлалыг ашигласан. Судалгааны үр дүнд оюутнуудын суралцах үйл явцыг илүү бүтэцтэй, үр ашигтай болгох боломжтой дижитал платформын суурь шийдэл боловсруулагдсан болно.

\vspace{3mm}
\textit{Түлхүүр үгс:} Платформ, Веб систем, Мобайл аппликейшн, Флашкарт, Толь бичиг, Зайтай давталт, Боловсролын технологи

\vspace{10mm}

\begin{center}
\textbf{\large Abstract}
\end{center}

\begin{english}
This thesis aims to design and develop an integrated web and mobile platform called ``FutureHub'' to support students of Computer Science Department in self-directed learning, project development, research activities, documentation, and participation in competitions. The platform is intended to enhance students' professional knowledge and skills by providing a structured and centralized digital learning environment.

Students often encounter various challenges while studying professional subjects using Japanese textbooks, conducting projects, and carrying out research. These challenges include insufficient understanding of technical terminology, limited availability of high-quality Mongolian learning resources, fragmented and unclear information sources, and the absence of a unified system to systematically document and present their work.

To address these issues, the proposed system consists of two main components: a flashcard-based mobile application and a web platform. The mobile application is built upon the Spaced Repetition System (SRS) to improve long-term retention of vocabulary and technical terms. It also incorporates additional features such as OCR-based text extraction from images, a collaborative technical dictionary, and text-to-speech functionality that allows users to generate audio files from flashcards for repeated listening and review.

The web platform integrates multiple modules including blogs and articles, a collaborative dictionary, project development logs, discussions, contests, a contribution ranking system, and a profile-portfolio module. Through these features, students can centralize knowledge, collaborate with peers, document their academic progress, and build a structured digital portfolio to demonstrate their competencies.

The system was developed using architectural design modeling, database schema design, prototype implementation, and user-based evaluation methods. As a result, a foundational digital solution was established to support more structured, efficient, and outcome-oriented learning among students.

\vspace{3mm}
\textit{Keywords:} Platform, Web System, Mobile Application, Flashcard, Dictionary, Spaced Repetition System, Educational Technology
\end{english}

\vspace{10mm}

\begin{center}
\textbf{\Large \textjapanese{要旨}}
\end{center}

\begin{japanese}
本研究は、コンピューターサイエン工学科の学生を対象とした統合型Webおよびモバイルプラットフォーム「FutureHub」の設計・開発を目的とする。本プラットフォームは、自主学習、プロジェクト開発、研究活動、記録管理、コンテスト参加などを支援し、学生の専門的知識および技能の向上を促進するための体系的かつ集約的なデジタル学習環境を提供するものである。

学生は、専門科目の学習や日本語の教科書の活用、プロジェクトおよび研究活動を行う過程において、専門用語の理解不足、質の高いモンゴル語学習資料の不足、情報の分散・不明確さ、ならびに成果を体系的に記録・提示する統一的な環境の欠如といった課題に直面している。

これらの課題を解決するため、本システムはフラッシュカード型モバイルアプリケーションとWebプラットフォームの二つの主要コンポーネントから構成される。モバイルアプリケーションは、Spaced Repetition System(SRS)に基づき、語彙および専門用語の長期記憶定着を支援する。さらに、OCR技術による画像からの文字抽出機能、共同編集型専門用語辞書、Text-to-Speech機能による音声ファイル生成などの補助機能を備えている。

Webプラットフォームは、ブログ・記事投稿、共同辞書、プロジェクトログ、ディスカッション、コンテスト、貢献度ランキングシステム、プロフィール・ポートフォリオ機能など複数のモジュールを統合している。これにより、学生は知識を一元管理し、協働学習を行い、学習過程を記録するとともに、自身の能力を示す体系的なデジタルポートフォリオを構築することが可能となる。

本システムの開発にあたっては、アーキテクチャ設計、データベース設計、プロトタイプ実装、およびユーザー評価手法を用いた。その結果、学生の学習をより体系的かつ効率的に支援するための基盤的なデジタルソリューションを構築した。

\vspace{3mm}
\textit{キーワード:} プラットフォーム、Webシステム、モバイルアプリケーション、フラッシュカード、辞書、間隔反復法、教育工学
\end{japanese}

