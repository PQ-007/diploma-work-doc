\addtotoc{Хавсралт}

\chapter*{Хавсралт}

\section*{A. Системийн модулиуд ба үүргүүдийн товч жагсаалт}

FutureHub системийн ерөнхий бүтэц, оролцогчдын харилцан үйлчлэлийг Зураг~\ref{fig:futurehub-context}-т харуулсан. Үүнийг модуль-үйлдлийн түвшинд нэгтгэн үзүүлэхэд:

\begin{table}[H]
\centering
\caption{FutureHub системийн үндсэн модулиуд}
\label{tab:modules}
\small
\begin{tabular}{p{4cm}p{10cm}}
\toprule
\textbf{Модуль} & \textbf{Зорилго, үндсэн үйлдлүүд} \\
\midrule
Флашкарт (мобайл) & Карт үүсгэх/засах/устгах, давталт (SRS), OCR-аар карт үүсгэх, аудио давталт \\
Толь бичиг & Нэр томьёо нэмэх, хайх, тайлбар/орчуулга харах, чанарын хяналт \\
Нийтлэл/Блог & Контент бичих/унших, мэдлэг хуваалцах \\
Асуулт-хариулт & Асуулт асуух, хариулах, хэлэлцүүлэг өрнүүлэх \\
Төслийн тэмдэглэл & Ажлын явц баримтжуулах, портфолио бүрдүүлэх \\
Тэмцээн/Сургалт & Оролцоо нэмэгдүүлэх үйл ажиллагаа, даалгавар/оролцоо \\
Ранкинг & Хувь нэмэр, идэвхийн оноо/үнэлгээ, урамшууллын механизм \\
\bottomrule
\end{tabular}
\end{table}

\section*{B. Өгөгдлийн загварын жишээ (концепц түвшин)}

Доорх хүснэгтүүд нь FutureHub системийн өгөгдлийн сангийн \textit{концепц} түвшний жишээ загвар юм.

\begin{table}[H]
\centering
\caption{Флашкарт өгөгдлийн жишээ бүтэц}
\label{tab:flashcard_schema}
\small
\begin{tabular}{p{4cm}p{10cm}}
\toprule
\textbf{Талбар} & \textbf{Тайлбар} \\
\midrule
user\_id & Карт эзэмшигч хэрэглэгч \\
front & Урд талын текст (ж. япон нэр томьёо) \\
back & Ар талын тайлбар/орчуулга \\
example & Жишээ өгүүлбэр/тайлбар \\
next\_review\_at & Дараагийн давталтын огноо (SRS) \\
interval\_days & Давталтын интервал (өдрөөр) \\
ease\_factor & Давталтын коэффициент \\
created\_at & Үүсгэсэн огноо \\
\bottomrule
\end{tabular}
\end{table}

\section*{C. Зайтай давталтын (SRS) хуваарь шинэчлэх псевдо код}

\begin{algorithm}[H]
\caption{Флашкартын дараагийн давталт төлөвлөх (ерөнхий)}
\label{alg:srs_schedule}
\begin{algorithmic}[1]
\State \textbf{Require:} $\text{card}(\text{interval\_days}, \text{ease\_factor}), \text{rating} \in \{0,1,2,3\}$
\State \textbf{Ensure:} updated $\text{interval\_days, ease\_factor, next\_review\_at}$
\If{$\text{rating} \le 1$}
    \State $\text{interval\_days} \gets 1$
\Else
    \State $\text{ease\_factor} \gets \max(1.3, \text{ease\_factor} + 0.1 \times (\text{rating}-2))$
    \State $\text{interval\_days} \gets \max(1, \lfloor \text{interval\_days} \times \text{ease\_factor} \rfloor)$
\EndIf
\State $\text{next\_review\_at} \gets \text{today} + \text{interval\_days}$
\end{algorithmic}
\end{algorithm}

\section*{D. OCR-аар карт үүсгэх урсгалын жишээ}

\begin{lstlisting}[language={}, caption={OCR-аас флашкарт үүсгэх жишээ payload}, label={lst:ocr_payload}]
{
  "image": "<base64 or file reference>",
  "language": "jpn",
  "selectedText": "...",
  "card": {
    "front": "...",
    "back": "...",
    "tags": ["KUT", "Terminology"]
  }
}
\end{lstlisting}

\section*{E. Хэрэглэгчийн үүрэг ба эрхийн матриц (жишээ)}

\begin{table}[H]
\centering
\caption{Үүрэгт суурилсан эрхийн матриц (RBAC) жишээ}
\label{tab:rbac_matrix}
\small
\begin{tabular}{p{5cm}ccc}
\toprule
\textbf{Үйлдэл} & \textbf{Оюутан} & \textbf{Багш} & \textbf{Админ} \\
\midrule
Карт үүсгэх/давтах & \checkmark & \checkmark & \checkmark \\
Толь бичигт шинэ нэр томьёо нэмэх & \checkmark & \checkmark & \checkmark \\
Нийтлэл/асуулт нийтлэх & \checkmark & \checkmark & \checkmark \\
Сургалт/тэмцээн үүсгэх & -- & \checkmark & \checkmark \\
Хэрэглэгчийн эрх удирдах & -- & -- & \checkmark \\
Агуулга баталгаажуулах/устгах & -- & \checkmark & \checkmark \\
\bottomrule
\end{tabular}
\end{table}

\section*{F. API endpoint-уудын жишээ жагсаалт}

\begin{table}[H]
\centering
\caption{FutureHub REST API-ийн жишээ endpoint}
\label{tab:api_endpoints}
\small
\begin{tabular}{p{2.5cm}p{5cm}p{6.5cm}}
\toprule
\textbf{Method} & \textbf{Endpoint} & \textbf{Зорилго} \\
\midrule
POST & /api/auth/login & Нэвтрэх, токен авах \\
GET & /api/flashcards & Хэрэглэгчийн картуудыг авах \\
POST & /api/flashcards & Шинэ карт үүсгэх \\
PATCH & /api/flashcards/:id/review & Давталтын үр дүн илгээж SRS төлөв шинэчлэх \\
GET & /api/dictionary?q=... & Нэр томьёо хайх \\
POST & /api/articles & Нийтлэл үүсгэх \\
POST & /api/discussions & Асуулт/сэдэв нээх \\
GET & /api/rankings & Contribution/Competition ранкинг авах \\
POST & /api/ocr/extract & Зурагнаас текст таних \\
\bottomrule
\end{tabular}
\end{table}

\section*{G. Туршилтын кейсийн хураангуй жишээ}

\begin{table}[H]
\centering
\caption{Системийн үндсэн туршилтын кейсүүд (жишээ)}
\label{tab:test_cases_appendix}
\small
\begin{tabular}{p{5.5cm}p{4cm}p{4.5cm}}
\toprule
\textbf{Кейс} & \textbf{Хүлээгдэж буй үр дүн} & \textbf{Төлөв} \\
\midrule
Толь бичгээс флашкарт үүсгэх & Карт амжилттай хадгалагдах & Амжилттай \\
Флашкарт review хийх & interval, next\_review\_at шинэчлэгдэх & Амжилттай \\
OCR-аас текст таних & Танигдсан текст засварт харагдах & Амжилттай (засвартай) \\
Нийтлэл нийтлэх ба унших & Нийтлэл жагсаалт/дэлгэрэнгүйд харагдах & Амжилттай \\
Ranking шинэчлэх & Хувь нэмрийн оноо өсөх & Амжилттай \\
\bottomrule
\end{tabular}
\end{table}

\section*{H. Contribution ranking тооцооллын ерөнхий зарчим}

FutureHub системд ранкингийн оноог модулийн оролцоонд жин өгч нэгтгэх ерөнхий зарчмаар тооцож болно:

\[
Score = w_fF + w_dD + w_aA + w_qQ + w_pP
\]

энд:

\begin{itemize}
  \item $F$ — флашкартын оролцоо,
  \item $D$ — толь бичгийн оролцоо,
  \item $A$ — нийтлэл/блог оролцоо,
  \item $Q$ — асуулт-хариултын оролцоо,
  \item $P$ — төслийн тэмдэглэлийн оролцоо,
  \item $w_*$ — тухайн оролцооны жин.
\end{itemize}

Энэхүү томьёо нь бодит системд чанарын шалгуур (баталгаажсан агуулга, эерэг үнэлгээ, админы хяналт)-тай хамт хэрэглэгдэх ёстой.