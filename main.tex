\documentclass[11pt, a4paper]{article}

% --- Журмын 3.1-д заасан тохиргоо [cite: 43] ---
% Үсгийн фонт: Times New Roman
% Үсгийн өндөр: 11pt (documentclass дээр тохируулсан)
% Цаасны хэмжээ: A4
% Зүүн талаас: 2.5 см, Бусад талаас: 2.0 см

\usepackage{fontspec}
\usepackage{polyglossia}
\setmainlanguage{mongolian}
\setmainfont{Times New Roman}

\usepackage{geometry}
\geometry{
    left=2.5cm,
    right=2.0cm,
    top=2.0cm,
    bottom=2.0cm
}

% --- Мөр хоорондын зай: Single [cite: 43] ---
\usepackage{setspace}
\singlespacing

% --- Догол мөрний өмнө 6pt зай авах [cite: 43] ---
\usepackage{parskip}
\setlength{\parskip}{6pt} 
\setlength{\parindent}{0pt} % Догол мөрөөс эхлэхгүй, харин зай авна гэж ойлгов (Журамд "Бичлэгийн догол мөрөөс эхлэх хэсэг бүрийн өмнө 6pt" гэсэн тул)

% --- Бусад хэрэгцээт багцууд ---
\usepackage{graphicx}
\usepackage{titlesec}
\usepackage{caption}
\usepackage[hidelinks]{hyperref}
\usepackage{enumitem}

% --- Гарчгийн тохиргоо (Журмын дагуу Bold, Center/Left) ---
\titleformat{\section}{\bfseries\centering\large}{\thesection.}{0.5em}{}
\titleformat{\subsection}{\bfseries\normalsize}{\thesubsection.}{0.5em}{}

% --- Ном зүй (Name and Year System) [cite: 67, 112] ---
\usepackage[backend=biber, style=authoryear, sorting=nyt]{biblatex}
\addbibresource{references.bib}

\begin{document}

% ==========================================
% НҮҮР ХУУДАС (Хавсралт 2-ын дагуу) [cite: 179-189]
% ==========================================
\begin{titlepage}
    \begin{center}
        \large\textbf{ШИНЭ МОНГОЛ ТЕХНОЛОГИЙН КОЛЛЕЖ}\\
        \vspace{1cm}
        \large\textbf{КОМПЮЬТЕРИЙН УХААНЫ ТЭНХИМ}\\ % Энд тэнхимээ бичнэ
        \vspace{4cm}
        
        \Large\textbf{КУТ-ийн оюутнуудад зориулсан “Futurehub” хэрэглээний цогц платформ  хөгжүүлэлт}\\
        \vspace{1cm}
        \normalsize/ДИПЛОМЫН ТӨГСӨЛТИЙН АЖИЛ/\\
        
        \vspace{6cm}
        
        \begin{flushright}
            \begin{tabular}{ll}
                Удирдсан багш: & ................................. /Ц.Амарбат/ \\
                Гүйцэтгэсэн оюутан: & ................................. /А.Билгүүнтүшиг/ \\
                               &(Код: s21c097b)
            \end{tabular}
        \end{flushright}
        
        \vfill
        
        \textbf{Улаанбаатар хот}\\
        \textbf{2025 он}
    \end{center}
\end{titlepage}

% ==========================================
% ДОТОР НҮҮР (Хавсралт 3-ын дагуу) [cite: 190-197]
% ==========================================
\newpage
\thispagestyle{empty}
\begin{center}
    \large\textbf{ШИНЭ МОНГОЛ ТЕХНОЛОГИЙН КОЛЛЕЖ}\\
    \vspace{1cm}
    \large\textbf{КОМПЮЬТЕРИЙН УХААНЫ ТЭНХИМ}\\
    \vspace{3cm}
    
    \textbf{Төгсөлтийн судалгааны ажил (мэргэжлийн индекс)}\\
    \vspace{2cm}
    
    \Large\textbf{КУТ-ийн оюутнуудад зориулсан “Futurehub” хэрэглээний цогц платформ хөгжүүлэлт}\\
    
    \vspace{6cm}
    
    \begin{flushright}
        \begin{tabular}{ll}
            Гүйцэтгэгч: & ................................. /А.Билгүүнтүшиг/ \\
            Удирдагч: & ................................. /Ц.Амарбат/
        \end{tabular}
    \end{flushright}
    
    \vfill
    Улаанбаатар хот\\
    2025
\end{center}

\newpage
\pagenumbering{roman} % Ром тоогоор дугаарлах (i, ii, iii...)

% ==========================================
% ТОВЧЛОЛ (Хавсралт 4-ийн дагуу) [cite: 198-213]
% ==========================================
\section*{ТОВЧЛОЛ}
\addcontentsline{toc}{section}{Товчлол}

\textbf{Сэдвийн нэр (Монгол):} КУТ-ийн оюутнуудад зориулсан “Futurehub” хэрэглээний цогц платформ хөгжүүлэлт\\
\textbf{Topic (English):} Development of "Futurehub" Comprehensive Application Platform for Computer Science Students

\noindent\textbf{Оюутан:} А.Билгүүнтүшиг (s21c097b) \hfill \textbf{Удирдагч:} Ц.Амарбат (Магистр)

\vspace{0.5cm}

\textbf{Хураангуй:} Энд судалгааны ажлын зорилго, арга зүй, үр дүн, дүгнэлтийг багтаасан 150-250 үгтэй хураангуйг бичнэ[cite: 161].

\textbf{Түлхүүр үг:} \textit{түлхүүр үг 1, түлхүүр үг 2, түлхүүр үг 3} [cite: 177]

\subsection*{1. Удиртгал}
Төгсөлтийн судалгааны ажлын үндэслэл, ач холбогдол, зорилго зорилтыг энд бичнэ[cite: 209].

\subsection*{2. Судалгааны арга зүй}
Туршилтад хэрэглэгдэх материал, дээж, хэрэглэж буй арга зүйг товч бичнэ[cite: 210].

\subsection*{3. Судалгааны үр дүн}
Туршилтын үр дүнг тайлбарлан бичнэ. Зураг, хүснэгт оруулж болно[cite: 211].

\subsection*{4. Дүгнэлт}
Судалгааны ажлын үр дүнг нэгтгэн дүгнэж бичнэ[cite: 212].

\newpage

% ==========================================
% ГАРЧИГ БА ЖАГСААЛТУУД [cite: 116-124]
% ==========================================
\tableofcontents
\newpage
\listoffigures
\listoftables
\newpage

\pagenumbering{arabic} % Араб тоогоор дугаарлах (1, 2, 3...)

% ==========================================
% ҮНДСЭН БИЧВЭР (Журмын 3.2-ын дагуу) [cite: 49]
% ==========================================

\section{УДИРТГАЛ}

\subsection{Сэдвийн сонголтын үндэслэл}

Оюутнуудад суралцах, дадлага хийх, төсөл хөгжүүлэх, судалгаа хийх үеэр дараах төрлийн хүндрэлүүд нийтлэг тулгардаг. Үүнд:

\begin{enumerate}
    \item Мэргэжлийн хичээлүүд япон хэл дээрх ном, сурах бичгийг ашиглах үед ханз, үгийн сан хангалтгүй байх;
    \item Монгол хэл дээр тохирсон орчуулгатай, шаардлага хангасан агуулгатай нийтлэл, бичвэр хомс байх;
    \item Судалж буй хичээлүүдийн хоорондын уялдаа холбооны тайлбар, зөвлөгөө байхгүй;
    \item Судлах агуулгад шаардагдах мэдлэг, ашиглах хэрэгсэл, жишээ төсөл гэх мэт мэдээлэл тодорхой бус байх;
    \item Төсөл, судалгааны ажил хийхэд зөвлөгөө авах тохиромжтой хүн, ментор олдохгүй байх;
    \item Цаашдын карьераа төлөвлөхөд шаардлагатай зөвлөгөө, мэдээлэл ил бус байх.
\end{enumerate}
Эдгээр хүндрэлүүд нь суралцагчид цаг алдан төөрөх, хэт их бүтэцжүүлээгүй мэдээлэл дунд гацах тохиолдлыг үүсгэдэг.

\subsection{Судалгааны ажлын зорилго, зорилт}

Энэхүү системийн ерөнхий зорилго нь оюутнуудад блог, зөвлөгөө, асуултууд, флашкарт, номын мэдээлэл, мэргэжилийн орчуулгын толь бичиг гэх мэт олон төрлийн агуулгыг нэг дор төвлөрүүлж, суралцах, төсөл хийх, тэмцээн уралдаанд оролцох зэрэг ур чадвараа хөгжүүлэхэд дэмжлэг үзүүлэх цогц платформ болгоход оршино. Ингэснээр оюутнууд цаг алдаж төөрөхөөс сэргийлж, суралцах үйл явцыг илүү үр бүтээлтэй, хялбар болгох боломжтой.

зорилтууд:
\begin{itemize}
    \item Системийн бүтэц, дизайны загвар боловсруулах.
    \item Системийн Веб болон гар утасны аппликейшн  хөгжүүлж, системийн туршилтын хувилбарыг бүтээх
    \item Системийн Системийг хэрэглээнд туршиж, санал хүсэлт, туршилтын үр дүнг цуглуулж боловсруулах
    \item Системийн Мэргэжлийн япон монгол орчуулгын дижитал сан үүсгэн толь бичгийн модульд ашиглах
    \item Системийн Анхан шатны мэдлэг бүхий контент, сургалтуудыг бүтээж системдээ оруулах
\end{itemize}
\section{СУДАЛГААНЫ СЭДВИЙН ОНОЛ, ӨНӨӨГИЙН ТҮВШИН}

\subsection{Асуудлын үндэслэл ба онолын суурь}
Дипломын ажлын хүрээнд оюутны сурах явцыг дэмжих зорилгоор хэрэгжүүлж буй "FutureHub" платформ нь хэрэглэгч төвтэй дизайн, тоглоомжуулалт (Gamification), болон танин мэдэхүйн сэтгэл судлалын Spaced Repetition System (SRS) зэрэг онолын үндэслэлүүд дээр суурилсан.

\begin{itemize}
    \item SRS (Spaced Repetition System): Мэдээллийг мартахаас урьдчилан сэргийлж, давталтын хугацааг оновчтой тодорхойлох алгоритмыг ашигласнаар оюутны үгийн сан болон мэргэжлийн нэр томьёог тогтоох чадварыг дээшлүүлнэ.
    \item Тоглоомжуулалт:Платформд "Contribution ranking," "Kanji battle" зэрэг элементүүдийг нэвтрүүлж, оюутнуудыг идэвхтэйгээр мэдлэг хуваалцах, суралцах үйл явцыг сонирхортой болгоно.
\end{itemize}

\subsection{Өнөөгийн түвшин ба ололт дутагдал}
Одоогийн байдлаар зах зээл дээр мэргэжлийн үг цээжлэхэд зориулсан Anki, Quizlet зэрэг үр дүнтэй платформууд байгаа хэдий ч, тэдгээр нь Монгол хэл дээрх KOSEN-ийн сургалтын онцлогт тохирсон контент, төслийн менежментийн хэрэгсэл, зөвлөгөөг нэг дор цогцоор нь өгч чадахгүй байна. "FutureHub" нь эдгээр дутагдлыг нөхөж, оюутны хэрэгцээнд бүрэн нийцсэн нэгдсэн экосистемийг бий болгох зорилготой юм.

\section{СУДАЛГААНЫ АРГА ЗҮЙ}

\subsection{Системийн архитектур ба технологийн шийдэл}
"FutureHub" платформыг хөгжүүлэхдээ орчин үеийн, тогтвортой технологийн шийдлийг сонгов.

\begin{itemize}
    \item Архитектур: Веб (Next.js/Typescript) болон Мобайл (Flutter/Dart)-ийн хосолсон бүтцийг ашигласан. Веб хувилбар нь бүх цогц функцуудыг хамарч, Мобайл хувилбар нь флаш картны хэрэглээнд голчлон анхаарна.
    \item Backend ба Өгөгдлийн сан: Cloud-native шийдэл болох \textbf{Supabase} (PostgreSQL) -ийг ашиглан API, Authentication, Realtime Database, болон Storage-ийн асуудлыг нэг дор шийдсэн.
    \item Гол Алгоритм: Флашкартны модульд Spaced Repetition System (SRS)-ийг хэрэгжүүлэн, хэрэглэгчийн өмнөх гүйцэтгэлд суурилан дараагийн давталтын хугацааг автоматаар тооцоолно.
\end{itemize}

\subsection{Хөгжүүлэлт}

\begin{itemize}
  \item Төлөвлөлт ба Дизайн: Хэрэглэгчийн хэрэгцээг тодорхойлох, системийн ерөнхий төсөөлөл, функцын жагсаалт, UI/UX-ийн загвар гаргах.
  \item Backend хөгжүүлэлт: Supabase дээр өгөгдлийн загварчлал, SRS алгоритмын логик, үндсэн API endpoint-уудыг (Жишээ нь: `POST /flashcards`) хэрэгжүүлэх.
  \item Frontend хөгжүүлэлт: Веб болон Мобайл платформуудыг бие даан хөгжүүлэх, API-тай холбох.
  \item Туршилт ба Сайжруулалт: Оюутнуудыг хамруулан туршилт хийж, хэрэглэгчийн саналд үндэслэн давталттайгаар сайжруулах.
\end{itemize}
\section{СУДАЛГААНЫ ҮР ДҮН, ДҮГНЭЛТ}
Судалгаанаас гарсан тодорхой үр дүн, түүнд хийсэн шинжилгээ, эцсийн дүгнэлтүүд байна.

\subsection{Хүлээгдэж буй үр дүн}
Нэгдсэн платформ: Оюутнууд сургалтын болон төслийн мэдээллийг нэг цэгээс авах боломжтой болно.
Идэвхжүүлэлт: Тоглоомжуулсан болон Contribution Ranking систем нь оюутнуудыг контент үүсгэх, хуваалцахад идэвхтэй оролцоход түлхэц өгнө.

\section{СУДАЛГААНЫ ҮР ДҮН, ДҮГНЭЛТ}
Судалгаанаас гарсан тодорхой үр дүн, түүнд хийсэн шинжилгээ, эцсийн дүгнэлтүүд байна[cite: 64].



% ==========================================
% НОМ ЗҮЙ [cite: 65]
% ==========================================
% \newpage
% \printbibliography[title={НОМ ЗҮЙ}]
% \addcontentsline{toc}{section}{Ном зүй}

% ==========================================
% ХАВСРАЛТ [cite: 70]
% ==========================================
\newpage
\section*{ХАВСРАЛТ 1}
\addcontentsline{toc}{section}{Хавсралт 1}
Нэмэлт материал, зураг, хүснэгт, асуулгын хуудас зэргийг энд оруулна.

\end{document}