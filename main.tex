\documentclass[11pt]{Thesis}
\usepackage{fontspec}
\setmainfont{Times New Roman}
\setsansfont{Arial}
\setmonofont[Scale=0.9]{Consolas}

% 1. LANGUAGE SETTINGS (Must come before biblatex)
\usepackage{polyglossia}
\setdefaultlanguage{mongolian}
\setotherlanguage{english}
\setotherlanguage{japanese}

% Гол текст (Кирилл) — Times New Roman хэвээр
\setmainfont{Times New Roman}

% Япон текстэнд CJK фонт
% (Windows дээр ихэвчлэн байдаг: Yu Mincho / Yu Gothic)
\newfontfamily\japanesefont{Yu Mincho}
\newfontfamily\japanesefontsf{Yu Gothic}


% Fixes the "Mongolian not supported" biblatex warning

\usepackage{enumitem}
\usepackage{wrapfig}
\usepackage{multirow}
\usepackage{indentfirst}
\usepackage{multicol}
\usepackage{listings}
\usepackage{tikz}
\usepackage{pgfplots}
\pgfplotsset{compat=1.18}

\begin{document}
\frenchspacing
\pagenumbering{roman}

\newcommand{\universityname}{ШИНЭ МОНГОЛ ТЕХНОЛОГИЙН КОЛЛЕЖ}
\newcommand{\departmentname}{КОМПЬЮТЕРЫН УХААНЫ ТЭНХИМ}
\newcommand{\titlename}{КУТ-ийн оюутнуудад зориулсан "FutureHub" хэрэглээний цогц платформ хөгжүүлэлт} 
\newcommand{\authorsname}{Амарсайхан Билгүүнтүшиг}
\newcommand{\authorShort}{А.БИЛГҮҮНТҮШИГ}
\newcommand{\studentID}{s21c097b}
\newcommand{\supervisorname}{Б.Батчулуун}
\newcommand{\cityname}{Улаанбаатар хот}
\newcommand{\gradYear}{2026}
\input{components/CoverPage}
\input{components/TitlePage}
\addtotoc{Хураангуй}

\vspace{10mm}

\begin{center}
\textbf{\large Хураангуй}
\end{center}

\sloppy

Хураангуйгаа энд бичнэ.

\vspace{3mm}
\textit{Түлхүүр үгс:} түлхүүр үгсээ энд бичнэ

\vspace{10mm}

\begin{center}
\textbf{\large Abstract}
\end{center}

\begin{english}
Write your abstract here.

\vspace{3mm}
\textit{Keywords:} add keywords here
\end{english}


\newpage
\tableofcontents

\input{components/Acrolist}

\newpage
\listoftab

\newpage
\listoffig
\input{components/Plan}
\newpage


\pagenumbering{arabic}
\setcounter{page}{1}


\chapter{Удиртгал}
\label{ch:introduction}

\section{Судалгааны үндэслэл}
КУТ-ийн оюутнууд мэргэжлийн хичээл судлах, япон хэл дээрх сурах бичиг болон эх сурвалж ашиглах, багийн төсөл хэрэгжүүлэх, судалгааны ажил хийх явцдаа дараах нийтлэг хүндрэлүүдтэй тулгардаг. Үүнд: мэргэжлийн нэр томьёоны ойлголт хангалтгүй байх, чанартай орчуулгатай агуулга хомс байх, мэдээллийн эх сурвалж тархай бутархай байх, мөн хийсэн ажлаа системтэй баримтжуулж ур чадвараа нотлох нэгдсэн орчин дутмаг байх зэрэг асуудал багтана.

Дээрх асуудлууд нь суралцах үйл явцыг тасалдуулах, бие даан ахих хурдыг бууруулах, хамтын суралцах боломжийг хязгаарлах сөрөг нөлөөтэй. Иймээс суралцах материал, нэр томьёоны сан, хамтын ажиллагаа, төсөл-суурьтай баримтжуулалт болон урамшууллын механизмыг нэгтгэсэн дижитал орчин хэрэгцээтэй байна.

\section{Асуудлын тодорхойлолт}
Одоогийн нөхцөлд оюутнууд:
\begin{itemize}
	\item Япон хэл дээрх мэргэжлийн нэр томьёог тогтоох, тогтмол давтах, давталтын хуваарь баримтлахад хүндрэлтэй;
	\item Орчуулга, тайлбар, жишээ, хэрэглээ зэрэг мэдээлэл нэг дор төвлөрөөгүй тул хайлт хийхэд хугацаа их зарцуулдаг;
	\item Төсөл, лабораторийн ажил, судалгааны ажлын явцыг нэг стандартаар баримтжуулах орчин дутмаг;
	\item Асуулт-хариулт, нийтлэл, блог, тэмцээн зэрэг хамтын орчны үйл ажиллагаа тархмал байдаг.
\end{itemize}

Эдгээрийг шийдвэрлэх зорилгоор “FutureHub” нэртэй веб болон мобайл цогц платформын загварчлал, хөгжүүлэлтийн арга зүйг судалж, прототип шийдэл боловсруулах шаардлага үүссэн.

\section{Судалгааны зорилго ба зорилтууд}
\subsection{Зорилго}
Энэхүү дипломын ажлын зорилго нь КУТ-ийн оюутнуудад зориулсан “FutureHub” нэртэй суралцах үйл явцыг дэмжих веб ба мобайл цогц платформын шийдэл боловсруулахад оршино.

\subsection{Зорилтууд}
Судалгааны зорилгыг хангахын тулд дараах зорилтуудыг дэвшүүлэв:
\begin{enumerate}
	\item Системийн хэрэглэгчдийн хэрэгцээ, асуудлыг тодорхойлж функциональ болон функциональ бус шаардлагыг боловсруулах;
	\item Флашкарт дээр суурилсан үг, нэр томьёо тогтоолтын модулийг зайтай давталтын (SRS) зарчимд тулгуурлан төлөвлөх;
	\item OCR ашиглан зургаас текст таних, толь бичиг/нэр томьёоны санг дэмжих шийдлийг тодорхойлох;
	\item Нийтлэл, блог, хэлэлцүүлэг, төслийн тэмдэглэл, тэмцээн, сургалт, ранкинг зэрэг веб модулиудын интеграцчилсан загварыг боловсруулах;
	\item Архитектур, өгөгдлийн сангийн зохион байгуулалт, прототип хөгжүүлэлт, туршилт үнэлгээний ерөнхий аргачлалыг хэрэгжүүлэх.
\end{enumerate}

\section{Судалгааны хамрах хүрээ}
Энэхүү ажил нь боловсролын зориулалттай веб ба мобайл платформын \textit{суурь шийдэл} боловсруулахад төвлөрсөн. Системийн бүрэлдэхүүн хэсгүүдийг хэрэглэгчийн үүрэг (оюутан/хэрэглэгч, багш, админ) болон үндсэн хэрэглээний урсгал (унших, бичих, үүсгэх, хянах, оролцох) дээр тулгуурлан тодорхойлов.

\begin{figure}[H]
\centering
\IfFileExists{pictures/futurehub-context.png}{
	\includegraphics[width=0.95\linewidth]{pictures/futurehub-context.png}
}{
	\fbox{\parbox{0.92\linewidth}{\centering
	\vspace{3mm}
	\textbf{Зураг оруулаагүй байна.}\\
	Дараах файлыг \texttt{pictures/} хавтсанд байршуулна уу: \texttt{futurehub-context.png}\\
	(Веб сайт + Гар утасны апп-ийн ерөнхий бүтэц бүхий зураг)
	\vspace{3mm}
	}}
}
\caption{FutureHub системийн ерөнхий бүтэц (веб ба мобайл бүрэлдэхүүн)}
\label{fig:futurehub-context}
\end{figure}

\section{Ажлын ач холбогдол ба шинэлэг тал}
Санал болгож буй шийдэл нь нэг платформ дээр нэр томьёо тогтоолт (флашкарт/SRS), толь бичиг, OCR, мэдлэг хуваалцах нийтлэл/блог, асуулт-хариулт, төслийн явцын тэмдэглэл (project log), тэмцээн/сургалт болон хувь нэмрийн үнэлгээ (ранкинг) зэрэг боломжийг нэгтгэснээр суралцах үйл явцыг илүү системтэй болгож, хамтын суралцах орчинг дэмжинэ.

\section{Тайлангийн бүтэц}
Энэхүү тайлан нь таван бүлгээс бүрдэнэ. Нэгдүгээр бүлэгт асуудлын үндэслэл, зорилго, хамрах хүрээг тодорхойлно. Хоёрдугаар бүлэгт холбогдох онол, судалгааны ажлуудын тоймыг харуулна. Гуравдугаар бүлэгт систем боловсруулах арга зүй, загварчлал, архитектурын шийдлийг тайлбарлана. Дөрөвдүгээр бүлэгт хэрэгжүүлэлтийн үр дүн, туршилт үнэлгээний дүгнэлтийг нэгтгэнэ. Тавдугаар бүлэгт нийт дүгнэлт болон цаашдын хөгжүүлэлтийн чиглэлийг санал болгоно.

\chapter{Судалгааны сэдвийн онол, өнөөгийн түвшин}
\label{ch:literature}

\section{Боловсролын технологи ба дижитал суралцах орчин}
Сүүлийн жилүүдэд дижитал суралцах орчин нь зөвхөн контент түгээх хэрэгсэл бус, харин хэрэглэгчийн оролцоо, ахицын хяналт, хамтын ажиллагаа, урамшууллын механизм бүхий экосистем болж хөгжиж байна. Ийм орчин нь мэдээллийг нэг дор төвлөрүүлж, суралцагчийн өдөр тутмын дадал (habit) болон ахицыг системтэйгээр дэмжих шаардлагатай.

“FutureHub” нь (i) хувь хүний түвшний нэр томьёо тогтоолт, (ii) нийтлэг мэдлэгийн сан/орчуулгын нэгдсэн орчин, (iii) хамтын суралцах, (iv) төслийн баримтжуулалт болон ур чадварын портфолио бүрдүүлэх боломжийг нэг платформ дээр нэгтгэх санаанд тулгуурласан.

\section{Флашкарт ба зайтай давталтын (SRS) зарчим}
Флашкарт нь богино нэгж мэдээллийг давтан суралцах энгийн бөгөөд үр дүнтэй арга юм. Харин давталтыг санамсаргүй бус, тухайн хэрэглэгчийн мартах магадлалыг тооцож \textit{зайтай давталт}-аар төлөвлөх нь урт хугацааны тогтоолтыг нэмэгдүүлдэг. SRS (Spaced Repetition System) нь:
\begin{itemize}
	\item карт бүрийн төлөв (шинэ/сурч буй/баталгаажсан гэх мэт),
	\item хэрэглэгчийн хариултын чанар,
	\item давталтын интервал болон дараагийн давталтын огноо
\end{itemize}
гэсэн ойлголтуудад тулгуурлан картын хуваарийг динамикаар шинэчилдэг.

FutureHub-ийн мобайл хэсэг нь мэргэжлийн нэр томьёо, үг хэллэгийг тогтоох зорилготой тул SRS зарчимд суурилсан флашкарт модуль зайлшгүй шаардлагатай.

\section{OCR технологи ба боловсролын хэрэглээ}
OCR (Optical Character Recognition) буюу дүрсээс текст таних технологи нь хэвлэмэл болон зурагласан эхээс текстийг автоматаар ялган авах боломж олгодог. Суралцах орчинд OCR нь:
\begin{itemize}
	\item ном, гарын авлага, зураг бүхий эх сурвалжаас нэр томьёо хурдан ялган авах,
	\item флашкарт үүсгэх процессыг хялбарчлах,
	\item толь бичиг/орчуулгын санг хурдан баяжуулах
\end{itemize}
зэрэг давуу талтай.

\section{Толь бичиг ба хамтын мэдлэгийн сан}
Мэргэжлийн нэр томьёоны тайлбар, орчуулга, жишээ өгүүлбэр зэрэг мэдээлэл нь нэгтгэсэн сан хэлбэрээр төвлөрөх үед хайлт хийх хугацаа буурч, ойлголтын зөрүү багасдаг. Хамтын оролцоотой (collaborative) толь бичиг нь хэрэглэгчдийн хувь нэмрээр байнга шинэчлэгдэж, агуулгын чанарыг сайжруулах боломжтой боловч хяналт, баталгаажуулалтын дүрэм (moderation) зайлшгүй шаардлагатай.

\section{Хамтын суралцах орчин: нийтлэл, хэлэлцүүлэг, төслийн тэмдэглэл}
Нийтлэл/блог болон асуулт-хариултын хэсэг нь суралцагчид туршлагаа хуваалцах, асуудлыг хамтран шийдвэрлэх боломж олгодог. Төслийн тэмдэглэл (project log) нь:
\begin{itemize}
	\item ажлын явц, гарсан шийдэл, тулгарсан асуудлыг бичгээр үлдээх,
	\item ахицын дарааллыг харах,
	\item портфолио болгон ашиглах
\end{itemize}
давуу талтай тул FutureHub-ийн веб хэсгийн цөм боломжуудын нэг юм.

\section{Урамшууллын механизм: тэмцээн, сургалт, ранкинг}
Геймификацийн элементүүд (оноо, ранкинг, тэмцээн, шагнал гэх мэт) нь хэрэглэгчийн оролцоо, идэвхийг нэмэгдүүлэхэд ашиглагддаг. Гэвч зөв төлөвлөөгүй тохиолдолд “тоо хөөх” зан төлөв үүсэх эрсдэлтэй тул хувь нэмрийг үнэлэх шалгуур (чанар, баталгаажуулалт, хэрэглэгчийн үнэлгээ) болон админ/багшийн хяналтыг давхар хэрэгжүүлэх нь зүйтэй.

\section{Дүгнэлт}
Холбогдох онол, үзэл баримтлалуудыг нэгтгэн үзвэл FutureHub системийн амжилттай хэрэгжүүлэлт нь (i) SRS-д суурилсан тогтвортой давталтын механизм, (ii) OCR ба толь бичгийн интеграцчилсан урсгал, (iii) хамтын суралцах контентын бүтэц, (iv) урамшуулал ба хяналтын зөв тэнцвэр гэсэн дөрвөн чиглэлийн уялдаа холбооноос ихээхэн хамаарна.

\chapter{Судалгааны арга зүй}
\label{ch:methodology}

\section{Ерөнхий аргачлал}
Энэхүү ажлын хүрээнд “FutureHub” системийг боловсруулахдаа дараах үе шаттай аргачлалыг баримталсан:
\begin{enumerate}
	\item шаардлага тодорхойлох ба асуудал шинжилгээ;
	\item системийн загварчлал (үүрэг-орчин, хэрэглээний урсгал, өгөгдлийн загвар);
	\item архитектурын шийдэл боловсруулах;
	\item прототип хэрэгжүүлэлт (веб + мобайл);
	\item туршилт, үнэлгээ, сайжруулалт.
\end{enumerate}

\section{Шаардлага тодорхойлох}
Шаардлагыг тодорхойлохдоо дипломын ажлын зорилго, зорилтууд болон хэрэглэгчдийн үндсэн асуудлууд (нэр томьёо тогтоолт, орчуулга/тайлбарын хомсдол, мэдээллийн тархай байдал, баримтжуулалтын дутмаг байдал)-д тулгуурлан функциональ шаардлагуудыг бүлэглэв. Үүнд:
\begin{itemize}
	\item \textbf{Мобайл (флашкарт)}: карт үүсгэх/засах/устгах, давталтын хуваарь, аудио давталт, OCR-аар карт үүсгэх,
	\item \textbf{Веб (контент)}: нийтлэл/блог, асуулт-хариулт, толь бичиг, төслийн тэмдэглэл,
	\item \textbf{Веб (орчин)}: тэмцээн, сургалт, ранкинг, админ/багшийн хяналт.
\end{itemize}

\section{Системийн загварчлал}
Системийн оролцогч талуудыг гурван үндсэн үүргээр авч үзэв: \textit{хэрэглэгч (оюутан)}, \textit{багш}, \textit{админ}. Зураг~\ref{fig:futurehub-context}-т веб ба мобайл хэсгийн гол модулиуд болон оролцогчдын харилцан үйлчлэлийг ерөнхий байдлаар үзүүлсэн.

\subsection{Хэрэглээний урсгалын товч тайлбар}
Диаграммын агуулгыг үйл ажиллагааны түвшинд дараах байдлаар нэгтгэнэ:
\begin{itemize}
	\item \textbf{Нийтлэл / Блог}: хэрэглэгч нийтлэл унших, бичих; багш чанарын хяналт хийх боломжтой.
	\item \textbf{Толь бичиг}: нэр томьёо нэмэх/засах/хайх; админ хянах.
	\item \textbf{Флашкарт}: үг хэллэгийг үүсгэх, давтах, ахиц хянах; OCR ашиглан зурагнаас карт үүсгэх урсгал.
	\item \textbf{Асуулт-хариулт}: хэрэглэгч асуулт асуух, хариулах; багш чиглүүлэх.
	\item \textbf{Төсөл боловсруулах}: төслийн явцын тэмдэглэл үүсгэх, баримтжуулах; админ/багш хянах.
	\item \textbf{Тэмцээн, сургалт, ранкинг}: оролцоо, оноолт, үнэлгээний механизмууд.
\end{itemize}

\section{Архитектурын шийдэл}
FutureHub нь хэрэглэгчийн төхөөрөмж дээр ажиллах \textit{мобайл аппликейшн}, хөтөч дээр ажиллах \textit{веб интерфэйс}, мөн өгөгдөл, бизнес логикийг удирдах \textit{сервер талын бүрэлдэхүүн} гэсэн гурван үндсэн хэсэгтэй гэж төлөвлөв.

\begin{itemize}
	\item \textbf{Клиент тал}: мобайл апп (флашкарт, OCR урсгал), веб платформ (нийтлэл, толь, Q\&A, project log гэх мэт).
	\item \textbf{Сервер тал}: хэрэглэгчийн өгөгдөл, контент, үнэлгээ/ранкинг, эрхийн удирдлагыг нэг API давхаргаар хангана.
	\item \textbf{Өгөгдлийн сан}: хэрэглэгч, контент, толь бичиг, флашкарт, үйл ажиллагааны бүртгэл, үнэлгээ зэрэг өгөгдлийг бүтэцтэй хадгална.
\end{itemize}

\section{Өгөгдлийн сангийн загварчлал}
Өгөгдлийн сангийн загварчлалд дараах үндсэн объектуудыг (entity) тодорхойлж, хоорондын хамаарлыг төлөвлөв: \textit{User}, \textit{Role}, \textit{Article/Post}, \textit{DictionaryTerm}, \textit{Flashcard}, \textit{ReviewLog}, \textit{ProjectLog}, \textit{Question/Answer}, \textit{Contest/Training}, \textit{ContributionScore}. Энэ нь контентын модуль бүр бие даан хөгжих боломжтой байх, мөн нийт ранкинг/үнэлгээний системд нэгтгэгдэх боломжийг хангана.

\section{Туршилт ба үнэлгээ}
Туршилтыг дараах хүрээнд авч үзэв:
\begin{itemize}
	\item \textbf{Функциональ туршилт}: модуль бүрийн үндсэн CRUD үйлдэл, хэрэглээний урсгалын алдааг шалгах;
	\item \textbf{Хэрэглэгчийн туршлага (UX)}: флашкарт давталтын урсгал, хайлт, контент унших/бичихийн ойлгомжтой байдал;
	\item \textbf{Интеграцийн туршилт}: веб ба мобайл хэрэглээнээс нэг өгөгдлийн эх үүсвэр (сервер/API, өгөгдлийн сан)-тэй харилцах уялдаа.
\end{itemize}


\chapter{Судалгааны үр дүн}

Энд судалгааны үндсэн үр дүн, тоон болон чанарын шинжилгээг оруулна уу.

\chapter{Дүгнэлт}
\label{ch:conclusion}

\section{Ерөнхий дүгнэлт}
Энэхүү дипломын ажлаар КУТ-ийн оюутнуудад зориулсан “FutureHub” нэртэй суралцах үйл явцыг дэмжих веб болон мобайл цогц платформын суурь шийдэл боловсруулав. Системийн үндсэн санаа нь:
\begin{itemize}
	\item мэргэжлийн нэр томьёо, үг хэллэгийг зайтай давталтад тулгуурлан тогтоох (флашкарт/SRS),
	\item OCR болон толь бичгийн интеграцчилал ашиглан контент үүсгэх үйл явцыг хөнгөвчлөх,
	\item нийтлэл, асуулт-хариулт, төслийн тэмдэглэлээр дамжуулан мэдлэгээ хуваалцах ба хөгжлөө баримтжуулах,
	\item тэмцээн, сургалт, ранкингаар оролцоог нэмэгдүүлэх
\end{itemize}
гэсэн дөрвөн гол чиглэл дээр тогтсон.

Ажлын үр дүнд суралцах нөөцүүдийг нэг дор төвлөрүүлэх, хамтын суралцах орчныг дэмжих, мөн оюутны хувь хүний хөгжлийг баримтжуулах боломж бүхий суурь дижитал орчны загвар бүрдсэн.

\section{Цаашдын хөгжүүлэлтийн чиглэл}
Цаашид системийг дараах чиглэлээр өргөжүүлэх боломжтой:
\begin{itemize}
	\item \textbf{Агуулгын чанарын хяналт}: багш/админы баталгаажуулалт, санал/үнэлгээ, давхардал илрүүлэлт;
	\item \textbf{SRS-ийн оновчлол}: хэрэглэгчийн зан төлөвт суурилсан персонализаци, статистик хэмжилт;
	\item \textbf{OCR сайжруулалт}: хэлний дэмжлэг, алдааны засвар, текстээс автоматаар карт санал болгох;
	\item \textbf{Аналитик}: суралцах ахиц, контентын хэрэглээ, оролцооны тайлан;
	\item \textbf{Интеграц}: сургалтын дотоод системүүдтэй холболт, импорт/экспорт.
\end{itemize}



%---------
% Ном зүй
%---------
\newpage
\addcontentsline{toc}{chapter}{Ном зүй}


\begingroup
% If you want to force the title text:
\renewcommand{\bibname}{Ном зүй}
\begin{thebibliography}{9}

\bibitem{Cepeda2006}
Cepeda. N. J., Pashler. H., Vul. E., Wixted. J. T., Rohrer. D. \textbf{(2006)}.
\textit{Distributed Practice in Verbal Recall Tasks}.
Psychological Bulletin, 132(3), 354–380.

\bibitem{Deterding2011}
Deterding. S., Dixon. D., Khaled. R., Nacke. L. \textbf{(2011)}.
\textit{From Game Design Elements to Gamefulness: Defining Gamification}.
Proceedings of the 15th International Academic MindTrek Conference.

\bibitem{Piaget1970}
Piaget. J. \textbf{(1970)}.
\textit{Science of Education and the Psychology of the Child}.
New York: Orion Press.

\bibitem{Siemens2005}
Siemens. G. \textbf{(2005)}.
\textit{Connectivism: A Learning Theory for the Digital Age}.
International Journal of Instructional Technology and Distance Learning, 2(1).

\bibitem{Vygotsky1978}
Vygotsky. L. S. \textbf{(1978)}.
\textit{Mind in Society}.
Cambridge, MA: Harvard University Press.

\bibitem{Zimmerman2002}
Zimmerman. B. J. \textbf{(2002)}.
\textit{Becoming a Self-Regulated Learner: An Overview}.
Theory Into Practice, 41(2), 64–70.

\end{thebibliography}
\endgroup
\newpage
\input{components/Appendix}
\input{components/Graditude}

\end{document}