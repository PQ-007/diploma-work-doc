\chapter{Дүгнэлт}
\label{ch:conclusion}

\section{Ерөнхий дүгнэлт}
Энэхүү дипломын ажлаар КУТ-ийн оюутнуудад зориулсан “FutureHub” нэртэй суралцах үйл явцыг дэмжих веб болон мобайл цогц платформын суурь шийдэл боловсруулав. Системийн үндсэн санаа нь:
\begin{itemize}
	\item мэргэжлийн нэр томьёо, үг хэллэгийг зайтай давталтад тулгуурлан тогтоох (флашкарт/SRS),
	\item OCR болон толь бичгийн интеграцчилал ашиглан контент үүсгэх үйл явцыг хөнгөвчлөх,
	\item нийтлэл, асуулт-хариулт, төслийн тэмдэглэлээр дамжуулан мэдлэгээ хуваалцах ба хөгжлөө баримтжуулах,
	\item тэмцээн, сургалт, ранкингаар оролцоог нэмэгдүүлэх
\end{itemize}
гэсэн дөрвөн гол чиглэл дээр тогтсон.

Ажлын үр дүнд суралцах нөөцүүдийг нэг дор төвлөрүүлэх, хамтын суралцах орчныг дэмжих, мөн оюутны хувь хүний хөгжлийг баримтжуулах боломж бүхий суурь дижитал орчны загвар бүрдсэн.

\section{Цаашдын хөгжүүлэлтийн чиглэл}
Цаашид системийг дараах чиглэлээр өргөжүүлэх боломжтой:
\begin{itemize}
	\item \textbf{Агуулгын чанарын хяналт}: багш/админы баталгаажуулалт, санал/үнэлгээ, давхардал илрүүлэлт;
	\item \textbf{SRS-ийн оновчлол}: хэрэглэгчийн зан төлөвт суурилсан персонализаци, статистик хэмжилт;
	\item \textbf{OCR сайжруулалт}: хэлний дэмжлэг, алдааны засвар, текстээс автоматаар карт санал болгох;
	\item \textbf{Аналитик}: суралцах ахиц, контентын хэрэглээ, оролцооны тайлан;
	\item \textbf{Интеграц}: сургалтын дотоод системүүдтэй холболт, импорт/экспорт.
\end{itemize}

