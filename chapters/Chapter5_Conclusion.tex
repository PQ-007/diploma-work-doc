\chapter{Дүгнэлт}
\label{ch:conclusion}

\section{Ерөнхий дүгнэлт}
Энэхүү дипломын ажлаар КУТ-ийн оюутнуудад зориулсан “FutureHub” нэртэй суралцах үйл явцыг дэмжих веб болон мобайл цогц платформын суурь шийдэл боловсруулав. Судалгааны явцад онолын үндэслэл, төстэй системүүдийн харьцуулсан шинжилгээ, SDLC арга зүйд суурилсан хэрэгжилт, туршилт-үнэлгээний үр дүнг нэг шугамд уялдуулан авч үзлээ.

Системийн үндсэн үзэл баримтлал нь дараах дөрвөн цөм чиглэлд төвлөрсөн:

\begin{itemize}
	\item мэргэжлийн нэр томьёо, үг хэллэгийг зайтай давталтад тулгуурлан тогтоох (флашкарт/SRS),
	\item OCR болон толь бичгийн интеграцчилал ашиглан контент үүсгэх үйл явцыг хөнгөвчлөх,
	\item нийтлэл, асуулт-хариулт, төслийн тэмдэглэлээр дамжуулан мэдлэгээ хуваалцах ба хөгжлөө баримтжуулах,
	\item тэмцээн, сургалт, ранкингаар оролцоог нэмэгдүүлэх.
\end{itemize}

Ажлын үр дүнд суралцах нөөцүүдийг нэг дор төвлөрүүлэх, хамтын суралцах орчныг дэмжих, мөн оюутны хувь хүний хөгжлийг баримтжуулах боломж бүхий суурь дижитал орчны загвар бүрдсэн.

\section{Судалгааны зорилт биелэлтийн дүгнэлт}

Нэгдүгээр бүлэгт дэвшүүлсэн зорилтуудын биелэлтийг дараах байдлаар нэгтгэж болно.

\begin{enumerate}[leftmargin=2em]
	\item \textbf{Шаардлагын тодорхойлолт}: хэрэглэгчийн гол асуудлуудыг функциональ, функциональ бус шаардлагад буулгаж тодорхойлсон;
	\item \textbf{SRS-д суурилсан модуль}: флашкартын давталтын адаптив логикийн суурь хэрэгжилт бий болсон;
	\item \textbf{OCR интеграц}: зурагнаас текст таних урсгалыг карт үүсгэх процессд уялдуулсан;
	\item \textbf{Веб модулийн интеграц}: нийтлэл, толь бичиг, хэлэлцүүлэг, төслийн тэмдэглэл, ранкингийг нэг системд холбосон;
	\item \textbf{Архитектур ба туршилт}: клиент–сервер бүтэц, API урсгал, модуль/интеграцийн туршилтын суурь аргачлал хэрэгжсэн.
\end{enumerate}

Иймээс энэхүү дипломын ажил нь анхны судалгааны зорилго болох “суралцах явцыг дэмжих нэгдсэн веб-мобайл платформын суурь шийдэл боловсруулах” зорилтыг хангаж чадсан гэж дүгнэв.

\section{Эрдэм шинжилгээний болон хэрэглээний ач холбогдол}

Энэхүү ажлын ач холбогдлыг хоёр түвшинд тодорхойлж болно.

\subsection{Эрдэм шинжилгээний ач холбогдол}

\begin{itemize}
	\item боловсролын технологийн онолууд (constructivism, social constructivism, self-regulated learning, gamification, SRS)-ыг нэг платформын шийдэлд интеграцчилсан;
	\item SRS, OCR, хамтын мэдлэгийн сан, contribution ranking-ийг нэг хүрээнд авч үзсэн судалгааны суурь гаргасан;
	\item цаашдын тоон туршилт, хэрэглэгчийн зан төлөвийн судалгаанд ашиглах концепц, арга зүйн суурь тогтоосон.
\end{itemize}

\subsection{Хэрэглээний ач холбогдол}

\begin{itemize}
	\item оюутны өдөр тутмын суралцах урсгалыг системчлэх боломж бүрдүүлсэн;
	\item контент бүтээх болон дахин ашиглах экосистемийн суурь бий болгосон;
	\item багш, админ хэрэглэгчийн хяналт, зохион байгуулалтын орон зайг тодорхойлсон;
	\item профайл-портфолио, ахицын ил тод байдлыг дэмжих механизм эхлүүлсэн.
\end{itemize}

\section{Судалгааны хязгаарлалт}

Энэхүү ажлын хүрээнд дараах хязгаарлалт байгааг тодорхойлж байна.

\begin{itemize}
	\item өргөн хэрэглэгчийн тоон мэдээлэлд суурилсан статистик үнэлгээ хийгдээгүй;
	\item OCR-ийн нарийвчлал, SRS-ийн параметрийн оновчлолыг урт хугацаанд туршиж баталгаажуулаагүй;
	\item аюулгүй байдлын гүн аудит, өндөр ачааллын тест зэрэг үйлдвэрлэлийн түвшний шалгалтууд бүрэн хийгдээгүй;
	\item бүх дотоод сургалтын системтэй бүрэн интеграцчилал хийгдээгүй.
\end{itemize}

Эдгээр хязгаарлалт нь ажлын үндсэн үр дүнг үгүйсгэхгүй боловч дараагийн шатанд зайлшгүй гүнзгийрүүлэх шаардлагатай чиглэлүүдийг тодорхой харуулж байна.

\section{Цаашдын хөгжүүлэлтийн чиглэл}

Цаашид системийг богино, дунд, урт хугацааны төлөвлөлтөөр хөгжүүлэх боломжтой.

\subsection{Богино хугацаа (MVP+ шат)}

\begin{itemize}
	\item \textbf{Агуулгын чанарын хяналт}: багш/админы баталгаажуулалтын урсгал, стандарт шалгуур;
	\item \textbf{OCR сайжруулалт}: танилтын дараах засварын ухаалаг санал;
	\item \textbf{Ранкингийн тэнцвэр}: модуль бүрийн онооны жинг хэрэглээний шинжилгээнд тулгуурлан тохируулах.
\end{itemize}

\subsection{Дунд хугацаа (Өргөтгөлтийн шат)}

\begin{itemize}
	\item \textbf{SRS персонализаци}: хэрэглэгчийн давталтын зан төлөвт тохируулсан параметр;
	\item \textbf{Суралцах аналитик}: ахиц, гүйцэтгэл, оролцооны самбар;
	\item \textbf{Сургалтын модуль өргөтгөл}: хичээл, даалгавар, үнэлгээний уялдаа.
\end{itemize}

\subsection{Урт хугацаа (Институцийн шат)}

\begin{itemize}
	\item \textbf{Интеграц}: сургуулийн дотоод систем, дан нэвтрэлт (SSO), импорт/экспорт;
	\item \textbf{Аюулгүй байдал}: эрхийн нарийвчилсан бодлого, аудитын мөр, илүү өргөн аюулгүй байдлын үнэлгээ;
	\item \textbf{Масштаб}: өндөр хэрэглэгчийн ачаалалд зориулсан архитектурын оновчлол.
\end{itemize}

\section{Хэрэгжүүлэлтийн зөвлөмж}

Системийг бодит орчинд нэвтрүүлэхдээ дараах зөвлөмжийг баримтлах нь зүйтэй.

\begin{enumerate}[leftmargin=2em]
	\item эхний ээлжинд нэг тэнхим эсвэл нэг курс дээр пилот нэвтрүүлэлт хийх;
	\item багш, оюутны богино сургалт зохион байгуулж хэрэглээний стандарт тогтоох;
	\item эхний үе шатанд контентын чанарын зөвлөлийн баг томилж толь бичиг, нийтлэлийн хяналтыг тогтвортой явуулах;
	\item хэрэглээний лог, санал хүсэлт, ахицын өгөгдлийг тогтмол цуглуулж дараагийн итерацад тусгах.
\end{enumerate}

\section{Төгсгөлийн дүгнэлт}

Энэхүү дипломын ажил нь КУТ-ийн оюутнуудын суралцах үйл явцад бодитоор тулгардаг тархмал контент, давталтын системгүй байдал, хамтын орчны хязгаарлалт зэрэг асуудлыг нэгтгэсэн өнцгөөс авч үзэж, FutureHub нэртэй нэгдсэн веб-мобайл платформын суурь шийдлийг боловсрууллаа.

Судалгааны үр дүн нь энэхүү платформыг цаашид бодит хэрэглээнд өргөжүүлэх онол-арга зүйн болон техникийн баталгаатай эхлэл болж байна. Иймээс FutureHub нь зөвхөн нэг удаагийн прототип бус, харин оюутны суралцах, хамтран бүтээх, хөгжлөө баримтжуулах соёлыг тогтвортой дэмжих боломжтой хөгжүүлэлтийн суурь экосистем гэж дүгнэж байна.

