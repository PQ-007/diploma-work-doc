\chapter{Судалгааны үр дүн}
\label{ch:results}

\section{Хэрэгжүүлсэн шийдлийн ерөнхий дүр зураг}
Энэхүү дипломын ажлын үр дүнд “FutureHub” нэртэй веб болон мобайл цогц платформын суурь шийдэл (прототип) боловсруулагдав. Систем нь (i) нэр томьёо тогтоолтод чиглэсэн флашкарт мобайл апп, (ii) мэдлэг хуваалцах ба баримтжуулалтын веб орчин гэсэн хоёр үндсэн хэсгээс бүрдэнэ.

\section{Флашкарт мобайл хэсгийн боломжууд}
Мобайл хэсэг нь зайтай давталтын (SRS) зарчимд тулгуурлан дараах үндсэн урсгалыг дэмжинэ:
\begin{itemize}
	\item карт үүсгэх, засах, устгах;
	\item давталтын хуваарийн дагуу өдөр тутмын давталт хийх;
	\item OCR ашиглан зурагнаас текст ялган авч, карт үүсгэх процессыг хөнгөвчлөх;
	\item сонсож давтах (audio-based review) зэрэг нэмэлт дэмжлэг.
\end{itemize}

\section{Веб платформын боломжууд}
Веб платформын хувьд контентын болон орчны боломжуудыг дараах байдлаар нэгтгэв:
\begin{itemize}
	\item \textbf{Нийтлэл/блог}: мэдлэг хуваалцах, туршлага тэмдэглэх;
	\item \textbf{Толь бичиг}: мэргэжлийн нэр томьёоны тайлбар, орчуулга, жишээ;
	\item \textbf{Асуулт-хариулт}: асуудал шийдвэрлэхэд чиглэсэн хамтын орчин;
	\item \textbf{Төслийн тэмдэглэл}: төслийн хөгжүүлэлтийн алхмууд, шийдлүүдийг баримтжуулах;
	\item \textbf{Тэмцээн/сургалт}: оролцоог нэмэгдүүлэх, суралцах сэдэл өгөх;
	\item \textbf{Ранкинг}: хэрэглэгчийн хувь нэмэр, идэвхийг үнэлэх.
\end{itemize}

\section{Модулиудын харилцан уялдаа}
Зураг~\ref{fig:futurehub-context}-т үзүүлсэнчлэн хэрэглэгч нь веб дээрх контентын модулиудтай (унших/бичих/оролцох) болон мобайл флашкарттай (үүсгэх/унших/давтах) зэрэгцэн ажиллана. Админ болон багшийн үүрэг нь контентын чанарын хяналт, зохион байгуулалт, сургалт/тэмцээний бэлтгэл, ранкингийн дүрэм хэрэгжилт зэрэгт төвлөрнө.

\section{Туршилт, үнэлгээний дүгнэлт}
Прототип шийдлийн туршилтын хүрээнд үндсэн хэрэглээний урсгалууд (контент нэмэх/унших, толь бичиг хайх, флашкарт үүсгэх ба давтах, асуулт-хариулт, project log бичих) нь логик дарааллаар ажиллах боломжтойг баталгаажуулав. Мөн OCR-т суурилсан карт үүсгэх урсгал нь нэр томьёо оруулах хугацааг бууруулах боломжтойг ажиглав.

Энэхүү бүлэгт тоон үзүүлэлт, өргөн хүрээний хэрэглэгчийн судалгаа оруулаагүй бөгөөд цаашдын ажлаар хэрэглэгчийн үнэлгээ (асуулга, ярилцлага), ашиглалтын статистик, SRS-ийн үр нөлөөг хэмжих туршилтыг гүнзгийрүүлэх боломжтой.
