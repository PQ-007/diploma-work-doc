\chapter{Судалгааны үр дүн}
\label{ch:results}

\section{Хэрэгжүүлсэн шийдлийн ерөнхий дүр зураг}

Энэхүү дипломын ажлын үр дүнд “FutureHub” нэртэй веб болон мобайл цогц платформын суурь шийдэл (прототип) боловсруулагдав. Систем нь (i) нэр томьёо тогтоолтод чиглэсэн флашкарт мобайл апп, (ii) мэдлэг хуваалцах ба баримтжуулалтын веб орчин гэсэн хоёр үндсэн хэсгээс бүрдэнэ.

Хэрэгжүүлэлтийн үр дүнг ерөнхийд нь дараах гурван түвшинд ангилж болно:

\begin{itemize}
	\item \textbf{Функцийн түвшин}: шаардлагад тусгасан үндсэн модулиуд ажиллах;
	\item \textbf{Интеграцийн түвшин}: веб-мобайл өгөгдөл API-гаар тууштай синхрончлогдох;
	\item \textbf{Хэрэглээний түвшин}: оюутны бодит суралцах урсгалд тохирох хэрэглээний сценариуд биелэх.
\end{itemize}

Эдгээр түвшинг үнэлэхдээ зөвхөн интерфейсийн харагдах байдлыг бус, өгөгдлийн урсгал, хэрэглэгчийн үүрэг, модулиудын уялдаа, ахицын тооцооллын логикийг хамтад нь авч үзсэн.

\section{Системийн бүрэлдэхүүн хэсгүүдийн хэрэгжилтийн түвшин}

FutureHub-ийн хөгжүүлэлт нь \textit{core module first} зарчмаар явагдсан. Энэ хүрээнд хамгийн түрүүнд суралцах үйл явцыг шууд дэмжих флашкарт, толь бичиг, контентын модулиудыг ажиллуулж, дараа нь оролцоог тогтворжуулах ранкинг, тэмцээн, сургалтын хэсгүүдийг нэгтгэсэн.

\begin{table}[htbp]
\centering
\caption{Модуль тус бүрийн хэрэгжилтийн нэгтгэл}
\begin{tabular}{|p{3.4cm}|p{3.4cm}|p{3.4cm}|p{3.4cm}|}
\hline
\textbf{Модуль} & \textbf{Гол боломж} & \textbf{Интеграц} & \textbf{Үр дүнгийн төлөв} \\ \hline
Flashcard & Карт үүсгэх, давтах, төлөв шинэчлэх & Dictionary, OCR, Ranking & Суурь урсгал тогтвортой \\ \hline
Dictionary & Үг/нэр томьёо нэмэх, хайх, засах & Flashcard, Article & Хайлт ба дахин ашиглалт ажилласан \\ \hline
Article/Blog & Markdown контент нийтлэх & Discussion, Profile & Хуваалцах, баримтжуулах боломж бүрдсэн \\ \hline
Discussion & Асуулт, хариулт, санал солилцох & Profile, Ranking & Хамтын оролцооны суурь бүрдсэн \\ \hline
Project Log & Төслийн явц тэмдэглэх & Profile, Ranking & Портфолио хэлбэрт шилжих боломжтой \\ \hline
Contest/Course & Оролцоо, хугацаа, бэлтгэл & Ranking, Teacher role & Идэвхжүүлэх орчин үүссэн \\ \hline
Ranking & Contribution оноо тооцох & Бүх модультай & Ахиц харагдахуйц болсон \\ \hline
\end{tabular}
\end{table}

\section{Флашкарт мобайл хэсгийн боломжууд}

Мобайл хэсэг нь зайтай давталтын (SRS) зарчимд тулгуурлан дараах үндсэн урсгалыг дэмжинэ:
\begin{itemize}
	\item карт үүсгэх, засах, устгах;
	\item давталтын хуваарийн дагуу өдөр тутмын давталт хийх;
	\item OCR ашиглан зурагнаас текст ялган авч, карт үүсгэх процессыг хөнгөвчлөх;
	\item сонсож давтах (audio-based review) зэрэг нэмэлт дэмжлэг.
\end{itemize}

Флашкарт модулийн гол үр дүн нь \textit{“гар аргаар цээжлэх”}-ээс \textit{“өгөгдөлд суурилсан давталт”} руу шилжсэн явдал юм. Хэрэглэгч карт давтах бүрт дараагийн огноо автоматаар шинэчлэгдэж, хүнд картууд богино мөчлөгөөр, хялбар картууд урт мөчлөгөөр гарч ирэх логик хэрэгжсэн.

\subsection{OCR-оос карт үүсгэх урсгалын дүн}

OCR урсгалын туршилтаар хэрэглэгч зурагнаас танигдсан текстийг шууд засварлаж, толь бичигтэй тулган, карт болгон хадгалах дараалал ажилласан. Энэ нь шинэ үг оруулах үеийн гарын ажиллагааг бууруулж, флашкарт үүсгэх ажиллагааг хэрэглэгчийн хувьд илүү хурдан болгосон.

Мөн OCR-оор орж ирсэн үгсийн заримд хэлний онцлогоос шалтгаалсан танилтын алдаа илэрсэн тул засварын интерфейс зайлшгүй шаардлагатайг баталсан.

\section{Веб платформын боломжууд}

Веб платформын хувьд контентын болон орчны боломжуудыг дараах байдлаар нэгтгэв:
\begin{itemize}
	\item \textbf{Нийтлэл/блог}: мэдлэг хуваалцах, туршлага тэмдэглэх;
	\item \textbf{Толь бичиг}: мэргэжлийн нэр томьёоны тайлбар, орчуулга, жишээ;
	\item \textbf{Асуулт-хариулт}: асуудал шийдвэрлэхэд чиглэсэн хамтын орчин;
	\item \textbf{Төслийн тэмдэглэл}: төслийн хөгжүүлэлтийн алхмууд, шийдлүүдийг баримтжуулах;
	\item \textbf{Тэмцээн/сургалт}: оролцоог нэмэгдүүлэх, суралцах сэдэл өгөх;
	\item \textbf{Ранкинг}: хэрэглэгчийн хувь нэмэр, идэвхийг үнэлэх.
\end{itemize}

Веб платформын хэрэгжилтээр оюутан зөвхөн мэдээлэл уншигч бус, мэдлэг бүтээгч болох урсгал бүрдсэн. Нийтлэл, асуулт-хариулт, төслийн тэмдэглэлийг нэг системд холбосноор хэрэглэгчийн хийсэн ажил тухайн мөчид хэрэглээтэй байхаас гадна дараагийн хэрэглэгчдэд дахин ашиглагдах мэдлэгийн хөрөнгө болж хувирч байна.

\subsection{Төслийн тэмдэглэл ба портфолио үнэ цэн}

Project log модуль нь “ямар үр дүнд хүрэв” гэдгээс гадна “яаж хүрэв” гэдгийг хадгалах боломж олгосон. Энэ нь:

\begin{itemize}
	\item багшид суралцагчийн бодит ахицыг үнэлэх,
	\item оюутанд өөрийн хөгжлийн түүхээ эргэн харах,
	\item багийн төслийн мэдлэгийг дараагийн гишүүдэд дамжуулах
\end{itemize}

боломжийг бий болгож, профайл-портфолио руу шууд уялдах суурь болсон.

\section{Модулиудын харилцан уялдаа}

Зураг~\ref{fig:futurehub-context}-т үзүүлсэнчлэн хэрэглэгч нь веб дээрх контентын модулиудтай (унших/бичих/оролцох) болон мобайл флашкарттай (үүсгэх/унших/давтах) зэрэгцэн ажиллана. Админ болон багшийн үүрэг нь контентын чанарын хяналт, зохион байгуулалт, сургалт/тэмцээний бэлтгэл, ранкингийн дүрэм хэрэгжилт зэрэгт төвлөрнө.

Модулиудын уялдааг дараах жишээ сценарийн дагуу баталгаажуулав:

\begin{enumerate}[leftmargin=2em]
	\item Оюутан нийтлэл унших явцдаа шинэ нэр томьёо олно;
	\item Тухайн нэр томьёог толь бичигт хайж утга, жишээг авна;
	\item Нэг товшилтоор флашкарт үүсгэж давталтдаа нэмнэ;
	\item Давталтын гүйцэтгэлээс үүдэн ранкингийн оноо шинэчлэгдэнэ;
	\item Идэвхтэй оролцсон хэрэглэгч тэмцээн/сургалтад оролцох урам зориг нэмэгдэнэ.
\end{enumerate}

Энэ нь FutureHub-ийн үндсэн үнэ цэнийг тодорхойлно: \textit{контентын хэрэглээ} ба \textit{суралцах ахиц} хооронд тасралтгүй өгөгдлийн холбоо үүсгэх.

\section{Хэрэглэгчийн үүргээр ангилсан үр дүн}

\subsection{Оюутан хэрэглэгчийн үр дүн}

Оюутны түвшинд дараах эерэг өөрчлөлт ажиглагдсан:

\begin{itemize}
	\item нэг дор олон төрлийн контент ашиглах боломж нэмэгдсэн;
	\item нэр томьёоны давталт тогтмолжих үндэс бүрдсэн;
	\item төслийн тэмдэглэл хөтлөх соёл, ахицын ил тод байдал сайжирсан.
\end{itemize}

\subsection{Багш хэрэглэгчийн үр дүн}

Багшийн түвшинд:

\begin{itemize}
	\item сургалтын материал, даалгаврын чиглэл, тэмцээний бэлтгэлийг нэг орчинд удирдах,
	\item суралцагчдын оролцоо, контент бүтээх идэвхийг ажиглах,
	\item толь бичиг, нийтлэл, хэлэлцүүлгийн чанарын хяналтыг үе шаттай хийх
\end{itemize}

боломжууд тодорхой болсон.

\subsection{Админ хэрэглэгчийн үр дүн}

Админы хувьд хэрэглэгчийн эрх, агуулгын зохистой байдал, нийт экосистемийн тогтвортой ажиллагаанд төвлөрөх хяналтын суурь бүрджээ.

\section{Туршилт, үнэлгээний дүгнэлт}

Прототип шийдлийн туршилтын хүрээнд үндсэн хэрэглээний урсгалууд (контент нэмэх/унших, толь бичиг хайх, флашкарт үүсгэх ба давтах, асуулт-хариулт, project log бичих) нь логик дарааллаар ажиллах боломжтойг баталгаажуулав. Мөн OCR-т суурилсан карт үүсгэх урсгал нь нэр томьёо оруулах хугацааг бууруулах боломжтойг ажиглав.

\subsection{Туршилтын кейсийн нэгтгэл}

\begin{table}[htbp]
\centering
\caption{Үндсэн туршилтын кейс ба ажиглалт}
\begin{tabular}{|p{5cm}|p{3cm}|p{6cm}|}
\hline
\textbf{Кейс} & \textbf{Төлөв} & \textbf{Ажиглалт} \\ \hline
Толь бичгээс карт үүсгэх & Амжилттай & Урсгал шууд, давхар оруулга багассан \\ \hline
OCR-оор үг таньж карт болгох & Амжилттай (засвартай) & Танилтын алдаа гарсан тохиолдолд гар засвар шаардлагатай \\ \hline
Нийтлэл нийтлээд хэлэлцүүлэг эхлүүлэх & Амжилттай & Хамтын оролцооны урсгал тасралтгүй \\ \hline
Project log бичээд профайлд харуулах & Амжилттай & Баримтжуулалт, портфолио холболт сайн \\ \hline
Contribution ranking шинэчлэх & Амжилттай & Үйлдэл бүр оноонд нөлөөлж, ахиц харагдсан \\ \hline
\end{tabular}
\end{table}

\subsection{Илэрсэн хязгаарлалт}

Туршилтын явцад дараах хязгаарлалтууд ажиглагдсан:

\begin{itemize}
	\item OCR-ийн нарийвчлал зурагны чанараас өндөр хамааралтай;
	\item ранкингийн онооны жинг илүү тэнцвэртэй болгох шаардлага бий;
	\item өргөн хэрэглэгчийн статистик цуглуулаагүй тул тоон баталгаа хязгаарлагдмал.
\end{itemize}

Эдгээр нь системийн суурь архитектурыг өөрчлөх түвшний асуудал бус, дараагийн итерацид оновчлох боломжтой сайжруулалтын чиглэлүүд гэж дүгнэв.

Энэхүү бүлэгт тоон үзүүлэлт, өргөн хүрээний хэрэглэгчийн судалгаа оруулаагүй бөгөөд цаашдын ажлаар хэрэглэгчийн үнэлгээ (асуулга, ярилцлага), ашиглалтын статистик, SRS-ийн үр нөлөөг хэмжих туршилтыг гүнзгийрүүлэх боломжтой.

\section{Бүлгийн дүгнэлт}

Судалгааны үр дүнгээс харахад FutureHub-ийн прототип нь анх дэвшүүлсэн зорилтуудтай уялдсан үндсэн урсгалуудыг хэрэгжүүлж чадсан. Ялангуяа флашкартын адаптив давталт, OCR-оос карт үүсгэх, веб талын мэдлэг хуваалцах модуль, contribution ranking-ийн суурь логик нь системийн гол ялгарал болох “суралцах + хамтын оролцоо + ахицын хяналт”-ыг бодитой болгож байна.

Нөгөө талаас өргөн хэмжээний статистик үнэлгээ, алгоритмын гүн оновчлол, чанарын баталгаажуулалтын автоматжуулалт нь цаашдын хөгжүүлэлтийн зайлшгүй үе шат болох нь тодорхой болов.