\chapter{Судалгааны сэдвийн онол, өнөөгийн түвшин}
\label{ch:literature}

\section{Боловсролын технологи ба дижитал суралцах орчин}
Сүүлийн жилүүдэд дижитал суралцах орчин нь зөвхөн контент түгээх хэрэгсэл бус, харин хэрэглэгчийн оролцоо, ахицын хяналт, хамтын ажиллагаа, урамшууллын механизм бүхий экосистем болж хөгжиж байна. Ийм орчин нь мэдээллийг нэг дор төвлөрүүлж, суралцагчийн өдөр тутмын дадал (habit) болон ахицыг системтэйгээр дэмжих шаардлагатай.

“FutureHub” нь (i) хувь хүний түвшний нэр томьёо тогтоолт, (ii) нийтлэг мэдлэгийн сан/орчуулгын нэгдсэн орчин, (iii) хамтын суралцах, (iv) төслийн баримтжуулалт болон ур чадварын портфолио бүрдүүлэх боломжийг нэг платформ дээр нэгтгэх санаанд тулгуурласан.

\section{Флашкарт ба зайтай давталтын (SRS) зарчим}
Флашкарт нь богино нэгж мэдээллийг давтан суралцах энгийн бөгөөд үр дүнтэй арга юм. Харин давталтыг санамсаргүй бус, тухайн хэрэглэгчийн мартах магадлалыг тооцож \textit{зайтай давталт}-аар төлөвлөх нь урт хугацааны тогтоолтыг нэмэгдүүлдэг. SRS (Spaced Repetition System) нь:
\begin{itemize}
	\item карт бүрийн төлөв (шинэ/сурч буй/баталгаажсан гэх мэт),
	\item хэрэглэгчийн хариултын чанар,
	\item давталтын интервал болон дараагийн давталтын огноо
\end{itemize}
гэсэн ойлголтуудад тулгуурлан картын хуваарийг динамикаар шинэчилдэг.

FutureHub-ийн мобайл хэсэг нь мэргэжлийн нэр томьёо, үг хэллэгийг тогтоох зорилготой тул SRS зарчимд суурилсан флашкарт модуль зайлшгүй шаардлагатай.

\section{OCR технологи ба боловсролын хэрэглээ}
OCR (Optical Character Recognition) буюу дүрсээс текст таних технологи нь хэвлэмэл болон зурагласан эхээс текстийг автоматаар ялган авах боломж олгодог. Суралцах орчинд OCR нь:
\begin{itemize}
	\item ном, гарын авлага, зураг бүхий эх сурвалжаас нэр томьёо хурдан ялган авах,
	\item флашкарт үүсгэх процессыг хялбарчлах,
	\item толь бичиг/орчуулгын санг хурдан баяжуулах
\end{itemize}
зэрэг давуу талтай.

\section{Толь бичиг ба хамтын мэдлэгийн сан}
Мэргэжлийн нэр томьёоны тайлбар, орчуулга, жишээ өгүүлбэр зэрэг мэдээлэл нь нэгтгэсэн сан хэлбэрээр төвлөрөх үед хайлт хийх хугацаа буурч, ойлголтын зөрүү багасдаг. Хамтын оролцоотой (collaborative) толь бичиг нь хэрэглэгчдийн хувь нэмрээр байнга шинэчлэгдэж, агуулгын чанарыг сайжруулах боломжтой боловч хяналт, баталгаажуулалтын дүрэм (moderation) зайлшгүй шаардлагатай.

\section{Хамтын суралцах орчин: нийтлэл, хэлэлцүүлэг, төслийн тэмдэглэл}
Нийтлэл/блог болон асуулт-хариултын хэсэг нь суралцагчид туршлагаа хуваалцах, асуудлыг хамтран шийдвэрлэх боломж олгодог. Төслийн тэмдэглэл (project log) нь:
\begin{itemize}
	\item ажлын явц, гарсан шийдэл, тулгарсан асуудлыг бичгээр үлдээх,
	\item ахицын дарааллыг харах,
	\item портфолио болгон ашиглах
\end{itemize}
давуу талтай тул FutureHub-ийн веб хэсгийн цөм боломжуудын нэг юм.

\section{Урамшууллын механизм: тэмцээн, сургалт, ранкинг}
Геймификацийн элементүүд (оноо, ранкинг, тэмцээн, шагнал гэх мэт) нь хэрэглэгчийн оролцоо, идэвхийг нэмэгдүүлэхэд ашиглагддаг. Гэвч зөв төлөвлөөгүй тохиолдолд “тоо хөөх” зан төлөв үүсэх эрсдэлтэй тул хувь нэмрийг үнэлэх шалгуур (чанар, баталгаажуулалт, хэрэглэгчийн үнэлгээ) болон админ/багшийн хяналтыг давхар хэрэгжүүлэх нь зүйтэй.

\section{Дүгнэлт}
Холбогдох онол, үзэл баримтлалуудыг нэгтгэн үзвэл FutureHub системийн амжилттай хэрэгжүүлэлт нь (i) SRS-д суурилсан тогтвортой давталтын механизм, (ii) OCR ба толь бичгийн интеграцчилсан урсгал, (iii) хамтын суралцах контентын бүтэц, (iv) урамшуулал ба хяналтын зөв тэнцвэр гэсэн дөрвөн чиглэлийн уялдаа холбооноос ихээхэн хамаарна.
