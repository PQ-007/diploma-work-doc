\chapter{Судалгааны сэдвийн онол, өнөөгийн түвшин}
\label{ch:literature}

\section{Боловсролын технологи ба дижитал суралцах орчин}

Сүүлийн жилүүдэд дижитал суралцах орчин нь зөвхөн контент түгээх хэрэгсэл бус, харин хэрэглэгчийн оролцоо, ахицын хяналт, хамтын ажиллагаа, урамшууллын механизм бүхий экосистем болж хөгжиж байна. Ийм систем мэдээллийг нэг дор төвлөрүүлж, суралцагчийн өдөр тутмын дадал болон ахицыг системтэйгээр дэмжих шаардлагатай байдаг \cite{Siemens2005}.

Дижитал суралцах орчны онолын үндэс нь боловсрол судлал болон мэдээллийн технологийн огтлолцолд бүрэлддэг. FutureHub-ийн үндсэн функцуудыг тодорхойлоход дараах онолууд суурь чиглэл болсон.

\begin{itemize}
    \item \textbf{Конструктивизм (Constructivism):} 
    Суралцагч нь мэдлэгийг идэвхтэйгээр бүтээдэг субъект гэж үздэг \cite{Piaget1970}. Энэ хүрээнд хэрэглэгчдэд өөрсдөө контент бүтээх, тайлбар бичих, жишээ нэмэх боломж олгох нь чухал.
    
    \item \textbf{Нийгмийн конструктивизм (Social Constructivism):} 
    Мэдлэг нь хамтын харилцаа, хэлэлцүүлгээр дамжин бүрэлддэг \cite{Vygotsky1978}. Тиймээс хэлэлцүүлгийн хэсэг, асуулт-хариулт, peer feedback механизмыг системийн цөмд оруулах шаардлага үүснэ.
    
    \item \textbf{Өөрийгөө зохицуулалттай суралцах (Self-Regulated Learning):} 
    Суралцагч зорилгоо тодорхойлох, ахицаа хянах, стратегиа өөрчлөх чадвартай байх ёстой \cite{Zimmerman2002}. Ахицын самбар, давталтын төлөв, ранкингийн үзүүлэлт нь энэ үйл явцыг дэмждэг.
    
    \item \textbf{Геймификацийн онол (Gamification):} 
    Оноо, түвшин, шагнал, ранкинг зэрэг элементүүд нь суралцах сэдлийг өсгөдөг \cite{Deterding2011}. Гэхдээ геймификаци нь зөвхөн “тоглоомчлол” биш, харин тогтвортой зан төлөв төлөвшүүлэх дизайны аргачлал гэж үзнэ.
    
    \item \textbf{Connectivism:}
    Мэдлэг нь зөвхөн хувь хүнд бус, харин хүмүүс, систем, эх сурвалжуудын холбоос дунд оршино \cite{Siemens2005}. Иймээс платформын зорилго нь “мэдлэг хадгалах” төдий бус, “мэдлэгийн урсгалыг холбох” байх ёстой.

    \item \textbf{Spaced Repetition Theory:}
    Хугацааны интервалтай давталт нь мартах муруйг багасгаж, урт хугацааны тогтоолтыг нэмэгдүүлдэг \cite{Cepeda2006}. Энэ онол нь FutureHub-ийн мобайл флашкартын цөм алгоритм болох шалтгаан болно.
\end{itemize}

Эдгээр онолын нэгтгэлээс харахад орчин үеийн суралцах платформ нь \textit{контент + давталт + хамтын оролцоо + урамшуулал + ахицын хяналт} гэсэн таван тулгуур бүрэлдэхүүнийг зэрэг агуулсан байх шаардлагатай.

\subsection{Суралцах орчны дизайны зарчим}

Онолын суурийг практикт буулгахын тулд дараах дизайны зарчмуудыг тодорхойлсон.

\begin{enumerate}[leftmargin=2em, itemsep=0.3em]
    \item \textbf{Нэг цэгийн хандалт:} нэр томьёо, нийтлэл, асуулт-хариулт, төслийн тэмдэглэлийг нэг экосистемд холбох;
    \item \textbf{Давталтын автомат удирдлага:} хэрэглэгчийн хариултад тулгуурлан дараагийн давталтыг тооцоолох;
    \item \textbf{Оролцоонд суурилсан үнэ цэн:} зөвхөн хэрэглээ бус, контент бүтээх оролцоог үнэлэх;
    \item \textbf{Веб-мобайл уялдаа:} оролцоо, ахиц, контентыг хоёр орчинд ижил өгөгдлөөр тэжээх;
    \item \textbf{Өргөтгөх боломж:} шинэ модуль (сургалт, тэмцээн, замнал) нэмэхэд архитектурын зөрчилгүй байх.
\end{enumerate}

\section{Төстэй системүүдийн судалгаа}

FutureHub системийн оновчтой архитектур, функцийн загварыг тодорхойлохын тулд Anki, Quizlet, Coursera, Udemy, GitHub, Codedex, Qiita, LeetCode, Stack Overflow зэрэг платформуудыг сонгон судалж, функцийн болон хэрэглэгчийн урсгалын түвшинд харьцуулсан шинжилгээ хийв.

\subsection{Флашкарт ба SRS-д суурилсан системүүд}

\textbf{Anki} нь Spaced Repetition System (SRS)-д суурилсан нээлттэй эхийн платформ бөгөөд картын давталтын интервал хэрэглэгчийн гүйцэтгэлд тулгуурлан шинэчлэгддэг. Давуу тал нь алгоритмын гүнзгий зохион байгуулалт ба өргөтгөх боломж; сул тал нь нийтлэг хэрэглэгчид зориулсан UX харьцангуй төвөгтэй.

\textbf{Quizlet} нь флашкарт, тест, тоглоомчилсон давталтын горим бүхий онлайн платформ бөгөөд контент хуваалцахад ээлтэй. Гэвч хувь хэрэглэгчийн гүнзгий персонализацитай SRS алгоритм нь Anki-тай харьцуулахад хязгаарлагдмал.

\subsection{Онлайн сургалтын платформууд}

\textbf{Coursera}, \textbf{Udemy}, \textbf{edX} зэрэг MOOC платформууд нь бүтэцтэй сургалтын туршлагыг санал болгодог. Видео лекц, тест, үнэлгээ, сертификат зэрэг бүрэлдэхүүн сайн хөгжсөн боловч дараах сул тал ажиглагдана:

\begin{itemize}
    \item микро-суралцах (micro-learning) урсгал өдөр тутмын түвшинд хангалтгүй,
    \item суралцагч өөрөө үүсгэсэн контентыг мэдлэгийн санд гүн нэгтгэх механизм сул,
    \item нэр томьёоны давталтыг тогтмол дэмжих SRS урсгал байхгүй.
\end{itemize}

\subsection{Хамтын мэдлэгийн сан ба хэлэлцүүлгийн системүүд}

\textbf{Wikipedia} нь хамтын мэдлэг бүтээх хүчирхэг загвар боловч суралцагч төвтэй ахицын хяналт, хувь хүний давталтын механизмгүй.

\textbf{Stack Overflow} нь асуулт-хариултад суурилсан хөгжүүлэгчдэд зориулсан платформ. Энэ платформ оролцоог идэвхжүүлэхэд амжилттай байдаг боловч системчилсэн суралцах замнал, толь бичиг, давталтын дэмжлэг дутмаг.

\subsection{Мэргэжлийн хөгжлийн платформууд}

\textbf{GitHub} нь кодны хувилбар хяналт, хамтран хөгжүүлэлтэд төвлөрдөг. Хамтын хөгжүүлэт хийж, кодоо хадгалахад хүчирхэг ч боловсролын домэйнд зориулсан сургалтын урсгалгүй. Тусдаа  GitHub Learn нь интерактив сургалт санал болгодог боловч нийт платформын нэг хэсэг болж интеграцчилсан биш.

\textbf{Codedex}, \textbf{Qiita} нь нийтлэлд суурилсан суралцах хэлбэрийг дэмждэг. Хэрэглэгч өөрийн мэдлэгийг бичиж нийтлэхэд давуу боловч нэр томьёоны нэгдсэн сан болон автомат давталттай нэгтгэл сул.

\textbf{LeetCode} нь алгоритмын сорилт, тэмцээн, ранкинг дээр хүчтэй. Гэвч ерөнхий мэдлэгийн сан, контентын олон төрөл, OCR-той толь бичиг зэрэг боловсролын өргөн экосистемийг бүрдүүлдэггүй.

\subsection{Харьцуулсан шинжилгээ}

\begin{table}[H]
\centering
\caption{Төстэй системүүдийн функцийн харьцуулалт (өргөтгөсөн)}
\begin{tabular}{|p{2.6cm}|c|c|c|c|c|p{3.3cm}|}
\hline
\textbf{Систем} & \textbf{SRS} & \textbf{Толь} & \textbf{Хамтын орчин} & \textbf{Ранкинг} & \textbf{OCR} & \textbf{Гол онцлог} \\ \hline
Anki & Тийм & Үгүй & Үгүй & Үгүй & Үгүй & Гүнзгий давталтын алгоритм \\ \hline
Quizlet & Хэсэгчлэн & Үгүй & Тийм & Хэсэгчлэн & Үгүй & Флашкарт + тоглоомчилол \\ \hline
Coursera & Үгүй & Үгүй & Хэсэгчлэн & Хэсэгчлэн & Үгүй & Бүтэцтэй онлайн сургалт \\ \hline
Stack Overflow & Үгүй & Үгүй & Тийм & Тийм & Үгүй & Q\&A ба reputation \\ \hline
LeetCode & Үгүй & Үгүй & Хэсэгчлэн & Тийм & Үгүй & Алгоритмын сорилт ба тэмцээн \\ \hline
FutureHub (зорилтот) & Тийм & Тийм & Тийм & Тийм & Тийм & Суралцах + хамтын мэдлэг + ахицын нэгдсэн орчин \\ \hline
\end{tabular}
\end{table}

\begin{figure}[htbp]
\centering
\includegraphics[width=0.9\textwidth]{pictures/Функцуудыг харуулсан диаграм.png}
\caption{App болон Website хэсгүүдийн функцийн огтлолцол}
\label{fig:function-venn}
\end{figure}

Зураг~\ref{fig:function-venn}-д FutureHub-ийн архитектурын үзэл баримтлал болох “мобайл дээр давталтын төвтэй, веб дээр контентын төвтэй, хоёр талд нийтлэг өгөгдлийн цөм” гэсэн зарчмыг харуулсан. Энэ огтлолцол нь хэрэглэгчийг нэг платформд түгжихгүй, харин тухайн нөхцөлдөө тохирсон орчноос үргэлжлүүлэн суралцах боломж олгоно.

\subsection{Нэгдсэн дүгнэлт}

Судалгаанаас харахад одоогийн системүүд тодорхой нэг чиглэлд (флашкарт, MOOC, асуулт-хариулт, сорилт, блог гэх мэт) гүнзгий боловч дараах элементүүдийг нэг дор нэгтгэсэн платформ хомс байна:

\begin{itemize}
    \item SRS-д суурилсан өдөр тутмын суралцах механизм,
    \item мэргэжлийн толь бичиг ба олон хэлний дэмжлэг,
    \item OCR-оор контент үүсгэх хөнгөн шийдэл,
    \item contribution-д суурилсан ранкинг,
    \item төсөл, нийтлэл, хэлэлцүүлэг, сургалт, тэмцээний нэгдсэн экосистем.
\end{itemize}

Иймд FutureHub-ийн судалгааны үнэ цэн нь “тодорхой нэг модулийг сайжруулах”-аас илүү “суралцах зан төлөвийг бүхэлд нь дэмжих интеграцчилсан орчин” боловсруулахад оршино.

\section{Хугацааны интервалт давталтын систем буюу Spaced Repetition System (SRS)}

Флашкарт нь богино нэгж мэдээллийг давтан суралцах энгийн бөгөөд үр дүнтэй арга. Харин давталтыг санамсаргүй бус, тухайн хэрэглэгчийн мартах магадлалыг тооцож хугацааны интервалтай төлөвлөхөд урт хугацааны тогтоолт мэдэгдэхүйц сайжирдаг \cite{Cepeda2006}.

SRS-ийн онолын суурь нь \textit{мартах муруй}-н ойлголт юм. Хэрэв хэрэглэгч мэдээллийг тодорхой хугацааны зайтай, зөв мөчид сэргээж байвал тухайн мэдээллийн урт хугацааны тогтоолт сайжирна. Иймд SRS алгоритм нь дараах өгөгдлийг карт бүр дээр хадгалдаг:

\begin{itemize}
    \item картын сүүлийн давталтын огноо,
    \item санах чанарын үнэлгээ (жишээ нь 0--5 шатлал),
    \item дараагийн давталтын хугацаа,
    \item хүнд/хялбарын коэффициент.
\end{itemize}

Практикт энэ логик нь “өдрийн хийх картын жагсаалт” хэлбэрээр хэрэглэгчид харагдаж, давталтын зохион байгуулалтын ачааллыг автоматаар бууруулдаг.
\subsection{SRS алгоритмын математик загвар}

Практикт өргөн хэрэглэгддэг Spaced Repetition алгоритм нь SuperMemo-2 (SM-2) загварт суурилдаг. Карт бүр дараах хувьсагчтай байна:

\begin{itemize}
    \item $I_n$ — $n$-дэх давталтын интервал (хоногоор),
    \item $EF$ — хялбаршлын коэффициент (Easiness Factor),
    \item $q$ — хэрэглэгчийн үнэлгээ ($0$--$5$),
    \item $n$ — давталтын тоо.
\end{itemize}

\subsubsection*{Интервал шинэчлэх}

\[
I_1 = 1
\]

\[
I_2 = 6
\]

\[
I_n = I_{n-1} \cdot EF \qquad (n \ge 3)
\]

Өөрөөр хэлбэл гурав дахь давталтаас эхлэн интервал өмнөх интервал дээр хялбаршлын коэффициентоор үржигдэнэ.

\subsubsection*{Хялбаршлын коэффициент шинэчлэх}

Хэрэглэгчийн үнэлгээ $q \in [0,5]$ үед:

\[
EF' = EF + \left(0.1 - (5 - q)\left(0.08 + (5 - q)\cdot 0.02\right)\right)
\]

Гэхдээ:

\[
EF' \ge 1.3
\]

Хэрэв $q < 3$ бол:

\begin{itemize}
    \item Давталтын тоо $n = 1$ болгож дахин эхлүүлнэ.
    \item Карт богино хугацаанд дахин давтагдана.
\end{itemize}

\subsubsection*{Мартах муруйн экспоненциал загвар}

SRS-ийн онолын үндэс нь мартах муруйн экспоненциал шинж чанарт тулгуурладаг:

\[
R(t) = e^{-t/S}
\]

энд:

\begin{itemize}
    \item $R(t)$ — тухайн хугацаанд мэдээллийг санах магадлал,
    \item $t$ — өнгөрсөн хугацаа,
    \item $S$ — тогтоолтын бат бөхийн параметр.
\end{itemize}

Энэхүү загвар нь хугацаа өнгөрөх тусам санах магадлал экспоненциал байдлаар буурдаг болохыг илэрхийлнэ. SRS алгоритм нь $R(t)$ огцом буурахаас өмнө давталтыг төлөвлөх замаар урт хугацааны тогтоолтыг нэмэгдүүлдэг.

\section{OCR технологи ба боловсролын хэрэглээ}

OCR (Optical Character Recognition) буюу дүрсээс текст таних технологи нь хэвлэмэл болон зурагласан эхээс текстийг автоматаар ялган авах боломж олгодог. Боловсролын орчинд OCR ашиглахын гол үнэ цэн нь гар аргаар шивэх хугацааг бууруулах, нэр томьёог алдаагүй оруулах, толь бичиг болон флашкарт руу шууд хөрвүүлэх боломж юм.

OCR-ийн хэрэглээг дараах үе шаттай авч үзэж болно:

\begin{enumerate}[leftmargin=2em]
    \item \textbf{Оролт}: ном, лекцийн слайд, тэмдэглэлийн зураг;
    \item \textbf{Урьдчилсан боловсруулалт}: гэрэлтүүлэг тэнцвэржүүлэх, тайрах, шуугиан бууруулах;
    \item \textbf{Танилт}: үсэг, үг, мөрийн танилт;
    \item \textbf{Дараах боловсруулалт}: зөв бичгийн засвар, хэл сонголт, нэр томьёоны баталгаажуулалт;
    \item \textbf{Интеграц}: толь бичиг болон флашкарт руу хадгалах.
\end{enumerate}

Ийм pipeline нь суралцагчийн контент үүсгэх үр ашгийг нэмэгдүүлж, гарын ажиллагааг багасгадаг боловч хэлний загварын нарийвчлал, фонтын төрөл, зурагны чанар зэрэг хүчин зүйлсэд мэдрэмтгий хэвээр байна.

\section{Толь бичиг ба хамтын мэдлэгийн сан}

Хамтын оролцоотой толь бичиг нь нийгмийн конструктивизмын үзэл баримтлалтай нийцэж, хэрэглэгчдийн хамтын оролцоогоор мэдлэгийг баяжуулах боломж олгодог \cite{Vygotsky1978}. Гэвч нээлттэй оролцооны давуу талын зэрэгцээ чанарын эрсдэл дагалддаг тул дараах удирдлагын зарчим шаардлагатай:

\begin{itemize}
    \item баталгаажсан болон баталгаажаагүй тайлбарын ялгаатай төлөв,
    \item багш/админы засвар, хяналтын урсгал,
    \item давхардал болон чанаргүй агуулга илрүүлэх механизм,
    \item эх сурвалж, жишээ, хэрэглээний талбарын стандарт.
\end{itemize}

FutureHub-ийн хүрээнд толь бичиг нь тусгаар модуль биш, харин флашкарт, нийтлэл, асуулт-хариулттай харилцан тэжээлцэх “мэдлэгийн цөм” гэж үзсэн.

\section{Урамшууллын механизм: тэмцээн, сургалт, ранкинг}

Геймификацийн элементүүд нь суралцах сэдэлд эерэг нөлөө үзүүлдэг болохыг судалгаагаар тогтоосон \cite{Deterding2011}. Гэвч оноо, шагнал нь зөвхөн идэвхийг өсгөхөөс гадна буруу сэдэл үүсгэх эрсдэлтэй тул ранкингийг дараах зарчмаар төлөвлөх нь зүйтэй:

\begin{itemize}
    \item \textbf{Оролцоо + чанарын хос үнэлгээ}: зөвхөн тоо бус хэрэглэгчийн санал, үнэлгээ, баталгаажуулалттай уях;
    \item \textbf{Модульд суурилсан оноо}: флашкарт, толь бичиг, нийтлэл, тусламжийн оролцоог ангилж тооцох;
    \item \textbf{Ил тод байдал}: хэрэглэгч оноо яаж өссөн шалтгааныг харах;
    \item \textbf{Сэдлийн тогтвортой байдал}: богино хугацааны тэмцээнийг урт хугацааны ахицын үзүүлэлттэй хослуулах.
\end{itemize}

Энэ зарчим нь “түр зуурын идэвх” бус “тогтвортой суралцах зан төлөв”-ийг дэмжихэд чиглэнэ.

\section{Судалгааны хоосон зай ба энэхүү ажлын байр суурь}

Холбогдох судалгаа, платформуудын шинжилгээнээс дараах хоосон зай тодорхой харагдсан:

\begin{enumerate}[leftmargin=2em]
    \item боловсролын орчинд SRS + OCR + толь бичгийн нэгдсэн шийдэл ховор,
    \item суралцах болон контент бүтээх урсгалыг веб-мобайл хосолсон загвараар нэгтгэсэн архитектур дутмаг,
    \item contribution ranking-ийг сургалтын чанарын үзүүлэлттэй уялдуулсан хэрэгжилт хязгаарлагдмал.
\end{enumerate}

Энэхүү дипломын ажил нь дээрх хоосон зайг нөхөх зорилгоор FutureHub-ийн концепцыг онол, функц, архитектур, арга зүйн түвшинд нэгтгэн тодорхойлж байгаагаараа практик ач холбогдолтой.

\section{Дүгнэлт}

Судалгааны сэдвийн онол, өнөөгийн түвшинг судлах явцад боловсролын технологи, дижитал суралцах орчин, SRS, OCR, толь бичиг, геймификацийн онол зэрэг олон талын хандлагыг нэгтгэн авч үзэв. Төстэй системүүдийн харьцуулсан шинжилгээ нь FutureHub системийн үндсэн ялгарал болох интеграцчилсан хандлагыг онолын болон хэрэглээний түвшинд баталгаажуулж байна.

Иймд дараагийн бүлэгт FutureHub системийг бодит хэрэгжилт рүү буулгах арга зүй, архитектур, технологийн сонголтыг үе шаттайгаар дэлгэрүүлэн тайлбарлана.