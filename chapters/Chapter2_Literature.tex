\chapter{Судалгааны сэдвийн онол, өнөөгийн түвшин}
\label{ch:literature}

\section{Боловсролын технологи ба дижитал суралцах орчин}

Сүүлийн жилүүдэд дижитал суралцах орчин нь зөвхөн контент түгээх хэрэгсэл бус, харин хэрэглэгчийн оролцоо, ахицын хяналт, хамтын ажиллагаа, урамшууллын механизм бүхий экосистем болж хөгжиж байна. Ийм систем мэдээллийг нэг дор төвлөрүүлж, суралцагчийн өдөр тутмын дадал болон ахицыг системтэйгээр дэмжих шаардлагатай байдаг \cite{Siemens2005}.

Дижитал суралцах орчны онолын үндэс нь боловсрол судлал болон мэдээллийн технологийн огтлолцолд бүрэлддэг. Үүнд дараах үндсэн онол, хандлагууд чухал байр суурь эзэлнэ.

\begin{itemize}
    \item \textbf{Конструктивизмын онол (Constructivism):} 
    Суралцагч нь мэдлэгийг идэвхтэйгээр бүтээдэг субъект гэж үздэг \cite{Piaget1970}. Дижитал орчин нь асуудал шийдвэрлэх, төсөл хэрэгжүүлэх, хэлэлцүүлэг өрнүүлэх замаар мэдлэгийг өөрөө бүтээх боломжийг олгох ёстой.
    
    \item \textbf{Нийгмийн конструктивизм (Social Constructivism):} 
    Мэдлэг нь хамтын харилцаа, хэлэлцүүлгээр дамжин бүрэлддэг \cite{Vygotsky1978}. Иймд хэлэлцүүлгийн форум, хамтын төсөл, peer review зэрэг функцууд нь суралцах үйл явцыг гүнзгийрүүлдэг.
    
    \item \textbf{Өөрийгөө зохицуулалттай суралцах онол (Self-Regulated Learning):} 
    Суралцагч өөрийн зорилгоо тодорхойлж, ахиц дэвшлээ хянаж, стратегиа өөрчилж чаддаг байх ёстой \cite{Zimmerman2002}. Ахиц гаргаж буй статистик, хэрэглэгчийн үнэлгээ зэрэг мэдээлэл нь энэ үйл явцыг дэмжинэ гэж үздэг онол.
    
    \item \textbf{Геймификацийн онол (Gamification Theory):} 
    Оноо, түвшин, шагнал, ранкинг зэрэг элементүүд нь суралцах сэдлийг нэмэгдүүлдэг \cite{Deterding2011}. Gamified review, contribution ranking зэрэг механизм нь идэвхийг тогтвортой хадгалах ач холбогдолтой.
    
    \item \textbf{Холбоо тогтоон хамтран суралцах онол (Connectivism):} 
    Энэ онол нь мэдлэгийг зөвхөн хүний тархинд хадгалагддаг зүйл биш, харин хүмүүс, мэдээллийн эх сурвалж, дижитал системүүдийн хоорондын холбоос (сүлжээ) дотор оршдог гэж үздэг. \cite{Siemens2005}. 

    \item \textbf{Хугацааны интервалт давталтын онол (Spaced Repetition Theory):} 
    Мэдлэгийг урт хугацаанд хадгалахад хугацааны зайтай давталт чухал үүрэгтэй \cite{Cepeda2006}. Алгоритмд суурилсан давталтын систем нь мартах муруйг багасгаж, тогтоолтыг сайжруулдаг.

\end{itemize}

Иймээс орчин үеийн боловсролын технологи нь зөвхөн агуулга хадгалах систем бус, харин суралцагчийн оролцоо, хамтын бүтээлч байдал, ахиц дэвшилийн хяналт, суралцах сэдэл зэргийг цогцоор нь дэмжих ухаалаг дижитал систем байх шаардлагатай юм.

\section{Төстэй системүүдийн судалгаа}

FutureHub системийн оновчтой архитектур, функцийн загварыг тодорхойлохын тулд боловсролын технологийн салбарт түгээмэл хэрэглэгддэг Anki, Quizlet, Coursera, Udemy, GitHub, Codedex, Qiita, LeetCode зэрэг платформуудыг сонгон судалж, функцийн болон онолын түвшинд харьцуулсан шинжилгээ хийв. 

Эдгээр системүүд нь флашкарт, онлайн сургалт, хамтын мэдлэгийн сан, асуулт-хариулт, геймификацийн механизм, ранкинг зэрэг өөр өөр чиглэлд мэргэшсэн шийдлүүдийг хэрэгжүүлсэн.

\subsection{Флашкарт ба SRS-д суурилсан системүүд}

\textbf{Anki} нь Spaced Repetition System (SRS)-д суурилсан нээлттэй эхийн флашкарт платформ бөгөөд давталтын интервал нь хэрэглэгчийн гүйцэтгэлд тулгуурлан алгоритмаар динамикаар шинэчлэгддэг. Энэ нь урт хугацааны ой тогтоолтыг нэмэгдүүлэх судалгааны үр дүнтэй нийцдэг \cite{Cepeda2006}. Гэвч хэрэглэгчийн интерфейс болон хамтын орчны боломж харьцангуй хязгаарлагдмал.

\textbf{Quizlet} нь флашкарт, тест, тоглоомчилсон давталтын горим бүхий онлайн платформ бөгөөд хэрэглэгчид контент үүсгэх, хуваалцах боломжтой. Геймификацийн элементүүд (оноо, leaderboard) ашигладаг \cite{Deterding2011}. 

\subsection{Онлайн сургалтын платформууд}

\textbf{Coursera}, \textbf{Udemy}, \textbf{edX} зэрэг MOOC платформууд нь их сургуулиуд болон мэргэжлийн байгууллагуудын боловсруулсан сургалтыг видео лекц, тест, үнэлгээ, сертификатын бүтэцтэйгээр хүргэдэг. Эдгээр нь конструктивизмын зарчимд нийцсэн бүтэцтэй сургалтын орчныг бүрдүүлдэг боловч:

\begin{itemize}
    \item микро-суралцах (micro-learning) болон өдөр тутмын давталтын систем сул,
    \item хэрэглэгчийн хувь нэмрийг мэдлэгийн санд шууд нэгтгэх механизм хязгаарлагдмал,
    \item SRS-д суурилсан урт хугацааны тогтоолтыг дэмжих функцгүй.
\end{itemize}

\subsection{Хамтын мэдлэгийн сан ба хэлэлцүүлгийн системүүд}

\textbf{Wikipedia} нь нийгмийн конструктивизмын тод жишээ бөгөөд хэрэглэгчид хамтран мэдлэг бүтээдэг \cite{Vygotsky1978}. Гэвч хувь хүний суралцах ахиц, давталтын систем байхгүй.

\textbf{Stack Overflow} нь асуулт-хариултын бүтэцтэй, contribution-д суурилсан ранкинг системтэй. Энэ нь геймификаци болон хамтын үнэлгээний механизмыг амжилттай хэрэгжүүлсэн боловч системчилсэн сургалтын замнал (learning pathway) үгүй.

\subsection{Мэргэжлийн контентын платформууд}

\textbf{GitHub} нь кодын хувилбар хянах, багийн хамтын ажиллагааг дэмжих хүчирхэг платформ боловч боловсролын контентын бүтэцтэй систем биш.

\textbf{Codedex} болон \textbf{Qiita} нь программистуудад зориулсан мэдлэгийн сан, нийтлэлийн платформууд бөгөөд кодын жишээ, тайлбар, хэлэлцүүлэг агуулдаг. Гэвч SRS-д суурилсан давталт болон суралцах ахицын хяналтын систем байхгүй.

\textbf{LeetCode} нь алгоритмын сорилтод төвлөрсөн платформ бөгөөд бодлогын сан, ранкинг, тэмцээний орчинтой. Гэвч мэдлэгийн нэгдсэн сан, толь бичиг, SRS механизм нэгтгэгдээгүй.

\subsection{Харьцуулсан шинжилгээ}

\begin{table}[H]
\centering
\caption{Төстэй системүүдийн функцийн харьцуулалт}
\begin{tabular}{|p{3cm}|c|c|c|c|p{3cm}|}
\hline
\textbf{Систем} & \textbf{SRS} & \textbf{Сургалт} & \textbf{Хамтын орчин} & \textbf{Геймификаци} & \textbf{Төв онцлог} \\ \hline
Anki & Тийм & Үгүй & Үгүй & Үгүй & Гүнзгий SRS алгоритм \\ \hline
Quizlet & Хязгаарлагдмал & Үгүй & Тийм & Тийм & Флашкарт + тоглоомчилол \\ \hline
Coursera & Үгүй & Тийм & Хэсэгчлэн & Хэсэгчлэн & Бүтэцтэй онлайн сургалт \\ \hline
Stack Overflow & Үгүй & Үгүй & Тийм & Тийм & Contribution ranking \\ \hline
LeetCode & Үгүй & Хэсэгчлэн & Хэсэгчлэн & Тийм & Алгоритмын сорилт \\ \hline

\end{tabular}
\end{table}

\subsection{Нэгдсэн дүгнэлт}

Судалгаанаас харахад одоогийн системүүд тодорхой нэг чиглэлд мэргэшсэн (флашкарт, MOOC, асуулт-хариулт, сорилт, блог гэх мэт) шийдэлтэй боловч дараах элементүүдийг нэг дор  хэрэгжүүлсэн платформ хомс байна:

\begin{itemize}
    \item SRS-д суурилсан суралцах механизм
    \item Хамтын мэдлэгийн сан ба хэлэлцүүлэг хийж санал бодлоо хуваалцах орчин
    \item Contribution-д суурилсан ранкинг систем
    \item Мэргэжлийн толь бичиг болон олон хэлний дэмжлэг
    \item Төсөл, блог, тэмцээний интеграцчилсан орчин
\end{itemize}

Иймд FutureHub систем нь дээрх платформуудын давуу талуудыг нэгтгэж, суралцах, бүтээх, хамтран ажиллах, ахиц дэвшлээ хэмжих боломжийг нэг дор төвлөрүүлсэн систем байх боломжтой гэж үзэж байна.

\section{Хугацааны интервалт давталтын систем буюу Spaced Repetition system (SRS)}

Флашкарт нь богино нэгж мэдээллийг давтан суралцах энгийн бөгөөд үр дүнтэй арга. Харин давталтыг санамсаргүй бус, тухайн хэрэглэгчийн мартах магадлалыг тооцож хугацааны интервалтай давталтаар төлөвлөх нь урт хугацааны тогтоолтыг нэмэгдүүлдэг \cite{Cepeda2006}. 

SRS (Spaced Repetition System) нь карт бүрийн төлөв, хэрэглэгчийн хариултын чанар, давталтын интервал зэрэг үзүүлэлтэд тулгуурлан давталтын хуваарийг динамикаар шинэчилдэг.

\section{OCR технологи ба боловсролын хэрэглээ}

OCR (Optical Character Recognition) буюу дүрсээс текст таних технологи нь хэвлэмэл болон зурагласан эхээс текстийг автоматаар ялган авах боломж олгодог. OCR нь боловсролын технологийн салбарт суралцах материалуудыг дижитал хэлбэрт хөрвүүлэх, текстийг хайх, индексжүүлэх, SRS-д суурилсан флашкарт үүсгэх зэрэг олон төрлийн хэрэглээтэй. OCR технологи нь суралцах материалуудыг илүү хүртээмжтэй, хайх боломжтой болгох ач холбогдолтой. Гэвч OCR-ийн нарийвчлал, хэлний дэмжлэг, гар бичмэл таних чадвар зэрэг хүчин зүйлүүд нь боловсролын хэрэглээнд нөлөөлдөг тул системийн дизайн болон хэрэглэгчийн туршлагыг сайтар төлөвлөх шаардлагатай болдог.

\section{Толь бичиг ба хамтын мэдлэгийн сан}

Хамтын оролцоотой толь бичиг нь нийгмийн конструктивизмын үзэл баримтлалтай нийцэж, хэрэглэгчдийн хамтын оролцоогоор мэдлэгийг баяжуулах боломж олгодог \cite{Vygotsky1978}. Ийм системд чанарын баталгаажуулалт, хяналтын механизм зайлшгүй шаардлагатай. Толь бичиг нь мэргэжлийн нэр томьёо, тодорхойлолт, жишээ, холбоотой ойлголтуудыг агуулж, хэрэглэгчид үүнийг ашиглан суралцах, судлах, төсөл хэрэгжүүлэх явцдаа ойлголтоо гүнзгийрүүлэхэд туслах ёстой.

\section{Урамшууллын механизм: тэмцээн, сургалт, ранкинг}

Геймификацийн элементүүд нь суралцах сэдэлд эерэг нөлөө үзүүлдэг болохыг судалгаагаар тогтоосон \cite{Deterding2011}. Гэвч чанарын хяналт, үнэлгээний зохистой тогтолцоог хавсарган хэрэгжүүлэх нь зүйтэй. Ялангуяа contribution-д суурилсан ранкинг системийг хөгжүүлэхэд хувь хүний хувь нэмрийг үнэлэх, хяналтын механизм, урамшууллын системийг цогцоор нь төлөвлөх шаардлагатай. Тэмцээн, сургалт, ранкинг зэрэг механизм нь хэрэглэгчдийн идэвхийг тогтвортой хадгалах, суралцах үйл явцыг үр дүнтэй болгох ач холбогдолтой.

\section{Дүгнэлт}

Судалгааны сэдвийн онол, өнөөгийн түвшинг судлах явцад боловсролын технологи, дижитал суралцах орчин, SRS, OCR, толь бичиг, геймификацийн онол зэрэг олон талын онол, хандлагуудыг авч үзсэн. Мөн төстэй системүүдийг функцийн болон онолын түвшинд харьцуулсан шинжилгээ хийж, одоогийн системүүдийн давуу болон сул талуудыг тодорхойлсон. Судалгааны үр дүнд FutureHub систем нь эдгээр онол, хандлагуудыг нэгтгэж, суралцах, бүтээх, хамтран ажиллах, ахиц дэвшлээ хэмжих боломжийг нэг дор төвлөрүүлсэн систем байх боломжтой гэж үзэж байна.