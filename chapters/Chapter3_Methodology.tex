\chapter{Судалгааны арга зүй}
\label{ch:methodology}

\section{Ерөнхий аргачлал}

Энэхүү дипломын ажлын хүрээнд КУТ-ийн оюутнуудад зориулсан “FutureHub” цогц веб болон мобайл платформыг боловсруулахдаа систем хөгжүүлэлтийн үе шатат аргачлал (System Development Life Cycle – SDLC)-ыг баримталсан. Судалгаа, шинжилгээ, загварчлал, хэрэгжилт, туршилт, сайжруулалтын дараах дарааллыг хэрэгжүүлэв:

\begin{enumerate}
	\item Шаардлагын тодорхойлолт боловсруулах;
	\item Системийн загварчлал хийх (use case, өгөгдлийн загвар, процессын урсгал);
	\item Архитектурын шийдэл боловсруулах;
	\item Технологийн сонголт хийх;
	\item Системийн UI/UX загвар дизайн боловсруулах;
	\item Хөгжүүлэлт хийх (веб болон мобайл);
	\item Туршилт, үнэлгээ, сайжруулалт хийх.
\end{enumerate}

\subsection{Шаардлагын тодорхойлолт боловсруулах}

Эхний үе шатанд хэрэглэгчийн хэрэгцээ шаардлагыг тодорхойлох зорилгоор асуудлын дүн шинжилгээ хийсэн. КУТ-ийн оюутнуудын суралцах явцад тулгардаг хүндрэлүүд (япон хэл дээрх нэр томьёо ойлгох, орчуулгын нэгдсэн сангүй байх, мэдээлэл тархмал байх, төсөл баримтжуулах орчин дутмаг гэх мэт)-ийг судалж, функциональ болон функциональ бус шаардлагуудыг тодорхойлов.
\subsubsection{Функциональ шаардлага}

\begin{itemize}
	\item Флашкарт үүсгэх, засах, хуваалцах;
	\item Толь бичиг ашиглах болон шинэ үг нэмэх;
	\item Блог, нийтлэл, төсөл нийтлэх;
	\item Хэлэлцүүлэг өрнүүлэх;
	\item Хэрэглэгчийн профайл болон хувь нэмрийн үнэлгээ тооцох;
	\item Олон хэлний дэмжлэг зэрэг орно.
\end{itemize}

\subsubsection{Функциональ бус шаардлага}
\begin{itemize}
	\item Өгөгдлийн аюулгүй байдал;
	\item Веб ба мобайл синхрончлол;
	\item Өргөтгөх боломжтой архитектур;
	\item Гүйцэтгэлийн хурд, найдвартай байдал;
\end{itemize}
зэрэг багтана.

\subsection{Системийн загварчлал}

Шаардлагын дагуу системийн логик бүтэц, хэрэглээний урсгал, өгөгдлийн бүтэц зэргийг UML болон ER диаграм ашиглан загварчилсан.

\begin{itemize}
	\item \textbf{Use Case диаграм:} Оюутан, багш, админ гэсэн үндсэн хэрэглэгчдийн үүргийг тодорхойлсон.
	\item \textbf{Процессын урсгал:} Флашкарт үүсгэх, давтах (Spaced Repetition алгоритм), хуваалцах зэрэг үндсэн үйл ажиллагааны дарааллыг боловсруулсан.
	\item \textbf{Өгөгдлийн загвар:} Хэрэглэгч, нийтлэл, флашкарт, толь бичиг, сэтгэгдэл, тэмцээн зэрэг entity-үүдийн хоорондын хамаарлыг ER диаграмаар тодорхойлсон.
\end{itemize}

\subsection{Архитектурын шийдэл}

Системийг клиент–сервер архитектур дээр суурилан боловсруулсан. Веб болон мобайл аппликейшн нь RESTful API-гаар дамжин backend сервертэй холбогдож, өгөгдлийг Supabase (PostgreSQL) өгөгдлийн санд хадгална.

Архитектурын үндсэн бүрэлдэхүүнүүд:
\begin{itemize}
	\item Front-end (Web + Mobile App);
	\item Backend API;
	\item PostgreSQL өгөгдлийн сан;
	\item Cloud storage (зураг, аудио файл хадгалах);
\end{itemize}

Мөн Spaced Repetition алгоритмыг backend түвшинд хэрэгжүүлж, хэрэглэгч бүрийн давталтын түүх дээр үндэслэн дараагийн харагдах картыг динамикаар тодорхойлдог байдлаар зохион байгуулсан.

\subsection{Технологийн сонголт}

Технологийн сонголтыг дараах шалгуурын дагуу хийсэн:
\begin{itemize}
	\item өргөтгөх боломж;
	\item хурд болон гүйцэтгэл;
	\item олон платформ дэмжих чадвар;
	\item хөгжүүлэлтийн үр ашиг.
\end{itemize}

Backend-д Supabase (PostgreSQL) ашигласан бол front-end талд веб болон мобайл орчинд тохирох framework сонгож, API-д суурилсан холболтыг хэрэгжүүлсэн.

\subsection{UI/UX загвар дизайн}

Хэрэглэгчдэд ээлтэй, ойлгомжтой интерфейс боловсруулах зорилгоор wireframe болон prototype загваруудыг боловсруулсан. Gamified review (Quizlet, Duolingo маягийн) механикийг ашиглан давталтыг илүү сонирхолтой болгох дизайн шийдлийг тусгасан.

\subsection{Хөгжүүлэлт}

Системийн хөгжүүлэлтийг модульчлагдсан байдлаар хийсэн. Үүнд:

\begin{itemize}
	\item Flashcard module;
	\item Dictionary module;
	\item Blog / Article module;
	\item Discussion module;
	\item Project log module;
	\item Contest module;
	\item Profile \& Contribution ranking module;
\end{itemize}

Веб болон мобайл хувилбарууд ижил backend API ашиглан өгөгдлийг синхрончлох байдлаар хэрэгжүүлсэн.

\subsection{Туршилт ба үнэлгээ}

Системийн туршилтыг дараах түвшинд хийсэн:
\begin{itemize}
	\item Unit test (тус бүрийн функц шалгах);
	\item Integration test (модуль хоорондын уялдаа);
	\item User acceptance test (хэрэглэгчийн туршилт).
\end{itemize}

Туршилтын үр дүнд илэрсэн алдаа, гүйцэтгэлийн асуудлуудыг засварлаж, UI болон алгоритмын сайжруулалт хийсэн.