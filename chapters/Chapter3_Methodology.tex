\chapter{Судалгааны арга зүй}
\label{ch:methodology}

\section{Ерөнхий аргачлал}

Энэхүү дипломын ажлын хүрээнд КУТ-ийн оюутнуудад зориулсан “FutureHub” цогц веб болон мобайл платформыг боловсруулахдаа систем хөгжүүлэлтийн үе шатат аргачлал (System Development Life Cycle – SDLC)-ыг баримталсан. Арга зүйн түвшинд \textit{шаардлага тодорхойлох → загварчлах → хэрэгжүүлэх → турших → сайжруулах}

Судалгаа, шинжилгээ, загварчлал, хэрэгжилт, туршилт, сайжруулалтын дараах дарааллыг хэрэгжүүлэв:

\begin{enumerate}
	\item Шаардлагын тодорхойлолт боловсруулах;
	\item Системийн загварчлал хийх (use case, өгөгдлийн загвар, процессын урсгал);
	\item Архитектурын шийдэл боловсруулах;
	\item Технологийн сонголт хийх;
	\item Системийн UI/UX загвар дизайн боловсруулах;
	\item Хөгжүүлэлт хийх (веб болон мобайл);
	\item Туршилт, үнэлгээ, сайжруулалт хийх.
\end{enumerate}

Энэхүү дараалал нь хатуу дараалсан (pure waterfall) биш, харин шаардлага болон UI/UX дээр эргэх холбоо үүсдэг \textit{итератив SDLC} зарчмаар явагдсан. Өөрөөр хэлбэл хэрэглэгчийн урсгал дээр илэрсэн зөрүүг дараагийн хөгжүүлэлтийн мөчлөгт буцаан оруулж, модуль тус бүрийг шат ахиулсан.

\subsection{Судалгааны объект, субъект}

\begin{itemize}
	\item \textbf{Судалгааны объект:} КУТ-ийн оюутнуудад зориулсан суралцах, контент бүтээх, ахиц үнэлэх дижитал орчны загвар;
	\item \textbf{Судалгааны субъект:} оюутан, багш, админ гэсэн хэрэглэгчийн гурван үндсэн бүлэг;
	\item \textbf{Судалгааны орчин:} веб болон мобайл клиент, REST API, Supabase/PostgreSQL өгөгдлийн сан.
\end{itemize}

\subsection{Судалгааны арга}

Системийн шийдлийг тодорхойлоход дараах аргуудыг хослуулан ашигласан.

\begin{enumerate}[leftmargin=2em]
	\item \textbf{Баримт бичгийн шинжилгээ:} ижил төрлийн платформуудын функц, архитектур, хэрэглэгчийн урсгал;
	\item \textbf{Харьцуулсан шинжилгээ:} FutureHub-ийн модуль бүрийг төстэй системтэй харьцуулж шаардлагын түвшинг тогтоох;
	\item \textbf{Загварчлалын арга:} use case, өгөгдлийн загвар, процессын урсгал;
	\item \textbf{Прототип хөгжүүлэлт:} суурь функцуудыг богино мөчлөгөөр хэрэгжүүлж баталгаажуулах;
	\item \textbf{Туршилт-үнэлгээ:} модуль, интеграц, хэрэглэгчийн урсгалын түвшинд чанарын шалгалт.
\end{enumerate}

\subsection{Шаардлагын тодорхойлолт боловсруулах}

Эхний үе шатанд хэрэглэгчийн хэрэгцээ шаардлагыг тодорхойлох зорилгоор асуудлын дүн шинжилгээ хийсэн. КУТ-ийн оюутнуудын суралцах явцад тулгардаг хүндрэлүүд (япон хэл дээрх нэр томьёо ойлгох, орчуулгын нэгдсэн сангүй байх, мэдээлэл тархмал байх, төсөл баримтжуулах орчин дутмаг гэх мэт)-ийг судалж, функциональ болон функциональ бус шаардлагуудыг тодорхойлов.

Шаардлагын тодорхойлолтыг боловсруулахдаа хэрэглэгчийн гол урсгал бүрийг \textit{(i) зорилго, (ii) оролт, (iii) үйлдэл, (iv) гаралт, (v) амжилтын шалгуур} гэсэн 5 хэмжээсээр задлан авч үзсэн. Ийм хэлбэрээр тодорхойлсноор модуль хоорондын уялдаа, давхардлыг эрт шатанд илрүүлэх боломжтой болсон.

\subsubsection{Функциональ шаардлага}

\begin{itemize}
  \item Зөвхөн өөрийн сургуулийн оюутан, оюутны кодоо ашиглан нэвтрэх
  \item Гадны этгээд хэрэглэгчээр бүртгүүлэхгүй, контентуудыг зөвхөн харах эрхтэй байх
  \item Зарим модуль нийтэд ил бус эсвэл хязгаарлагдмал байдлаар байх
  \item Report (bug, user), request feature гэсэн тусдаа сувгаар хэрэглэгч санал хүсэлтээ илгээдэг байх
	\item Флашкарт үүсгэх, засах, хуваалцах;
	\item Толь бичиг ашиглах болон шинэ үг нэмэх;
	\item Блог, нийтлэл, төсөл нийтлэх;
	\item Хэлэлцүүлэг өрнүүлэх;
	\item Хэрэглэгчийн профайл болон хувь нэмрийн үнэлгээ тооцох;
	\item Олон хэлний дэмжлэг зэрэг орно.
\end{itemize}

Функциональ шаардлагын хувьд \textit{core} болон \textit{extended} гэсэн хоёр ангиллаар авч үзсэн. Үүнд SRS давталт, толь бичиг, нийтлэл, асуулт-хариулт нь core шаардлагад; тэмцээн, сургалт, өргөтгөсөн аналитик нь extended шаардлагад хамаарна.

\subsubsection{Функциональ бус шаардлага}
\begin{itemize}
	\item Өгөгдлийн аюулгүй байдал;
	\item Веб ба мобайл синхрончлол;
	\item Өргөтгөх боломжтой архитектур;
	\item Гүйцэтгэлийн хурд, найдвартай байдал;
\end{itemize}
зэрэг багтана.

Мөн функциональ бус шаардлагыг дараах техникийн түвшний зорилтуудаар тодруулсан:

\begin{itemize}
	\item нэг хэрэглэгчийн энгийн асуулгын (query) хариу өгөх хугацааг практик хэрэглээнд мэдрэгдэхүйц сааталгүй байлгах;
	\item вэб ба мобайл дээр өгөгдлийн тасралтгүй байдлыг API-ийн нэг эх сурвалжаар хангах;
	\item контентын модуль нэмэгдэхэд өгөгдлийн загварыг дахин үндсээр нь эвдэхгүй өргөтгөх боломжтой байх.
\end{itemize}

\subsection{Системийн загварчлал}

Шаардлагын дагуу системийн логик бүтэц, хэрэглээний урсгал, өгөгдлийн бүтэц зэргийг UML болон ER диаграм ашиглан загварчилсан.

\begin{itemize}
	\item \textbf{Use Case диаграм:} Оюутан, багш, админ гэсэн үндсэн хэрэглэгчдийн үүргийг тодорхойлсон.
	\item \textbf{Процессын урсгал:} Флашкарт үүсгэх, давтах (Spaced Repetition алгоритм), хуваалцах зэрэг үндсэн үйл ажиллагааны дарааллыг боловсруулсан.
	\item \textbf{Өгөгдлийн загвар:} Хэрэглэгч, нийтлэл, флашкарт, толь бичиг, сэтгэгдэл, тэмцээн зэрэг entity-үүдийн хоорондын хамаарлыг ER диаграмаар тодорхойлсон.
\end{itemize}

\begin{figure}[htbp]
\centering
\includegraphics[width=0.92\textwidth]{pictures/futurehub-context.png}
\caption{FutureHub системийн хэрэглэгч, модуль, үүргийн контекст диаграм}
\label{fig:futurehub-context}
\end{figure}

Зураг~\ref{fig:futurehub-context}-т үзүүлснээр системийн төвд \textit{Хэрэглэгч} байрлаж, мобайл талд нэр томьёо суралцах урсгал (флашкарт, толь бичиг), веб талд бүтээмжийн урсгал (нийтлэл, төсөл, сургалт, тэмцээн, ранкинг) байрласан. Админ болон багш хэрэглэгч нь чанарын хяналт, агуулгын баталгаажуулалт, зохион байгуулалтын үүргийг гүйцэтгэхээр загварчилсан.

\subsubsection{Use Case-ийн тайлбар}

Use case загварчлалын хүрээнд хэрэглэгчийн гол үйлдлүүдийг дараах бүлгээр тодорхойлов.

\begin{itemize}
	\item \textbf{Оюутан}: бүртгүүлэх, нэвтрэх, карт үүсгэх, давтах, нийтлэл/асуулт нийтлэх, төслийн тэмдэглэл хөтлөх, сургалт/тэмцээнд оролцох;
	\item \textbf{Багш}: сургалт үүсгэх, агуулга бэлтгэх, чанарын хяналт хийх, зөвлөмж өгөх;
	\item \textbf{Админ}: хэрэглэгчийн эрх удирдах, агуулгын бодлого хэрэгжүүлэх, тайлан хянах.
\end{itemize}

\subsubsection{Өгөгдлийн загварын зарчим}

Өгөгдлийн загварыг төлөвлөхдөө дараах зарчмыг баримталсан:

\begin{enumerate}[leftmargin=2em]
	\item \textbf{Entity тусгаарлалт}: хэрэглэгч, контент, давталтын логик, оролцооны үнэлгээ тусдаа хүснэгтүүдэд;
	\item \textbf{Холбоосын ил тод байдал}: нийтлэл-толь бичиг, карт-толь бичиг, хэрэглэгч-контент холбоог гадаад түлхүүрээр тогтвортой хадгалах;
	\item \textbf{Өсөлтөд бэлэн байдал}: шинэ модуль нэмэхэд үндсэн хүснэгтийн бүтцийг эвдэхгүй өргөтгөх;
	\item \textbf{Аудит мөр}: контент үүсгэсэн, зассан, баталгаажуулсан хугацааны мэдээлэл хадгалах.
\end{enumerate}

\subsection{Архитектурын шийдэл}

Системийг клиент–сервер архитектур дээр суурилан боловсруулсан. Веб болон мобайл аппликейшн нь RESTful API-гаар дамжин backend сервертэй холбогдож, өгөгдлийг Supabase (PostgreSQL) өгөгдлийн санд хадгална.

Архитектурын үндсэн бүрэлдэхүүнүүд:
\begin{itemize}
	\item Front-end (Web + Mobile App);
	\item Backend API;
	\item PostgreSQL өгөгдлийн сан;
	\item Cloud storage (зураг, аудио файл хадгалах);
\end{itemize}

Мөн Spaced Repetition алгоритмыг backend түвшинд хэрэгжүүлж, хэрэглэгч бүрийн давталтын түүх дээр үндэслэн дараагийн харагдах картыг динамикаар тодорхойлдог байдлаар зохион байгуулсан.

\begin{figure}[htbp]
\centering
\includegraphics[width=0.98\textwidth]{pictures/Системийн архитектур.png}
\caption{FutureHub системийн хялбарчилсан клиент-сервер архитектур}
\label{fig:system-architecture}
\end{figure}

Зураг~\ref{fig:system-architecture}-т архитектурын мэдээллийн урсгалыг харуулсан. Клиентээс API хүсэлт ирж, backend нь өгөгдлийн (metadata) асуулгыг PostgreSQL-с, файл төрлийн агуулгыг cloud storage-оос тус тус шийдэн хэрэглэгчид буцаадаг. Ийм хуваарилалт нь мультимедиа контент ихсэх үед өгөгдлийн сангийн ачааллыг тусгаарлах давуу талтай.

\subsubsection{Архитектурын давуу тал}

\begin{itemize}
	\item веб ба мобайл дээр бизнес логикийг backend төвлөрүүлснээр давхардал буурна;
	\item өгөгдөл болон файлын урсгалыг тусгаарласнаар өргөтгөхөд хялбар;
	\item модуль нэмэхэд API endpoint нэмэх байдлаар шат ахиулж хөгжүүлэх боломжтой;
	\item хэрэглэгчийн ахиц, contribution зэрэг тооцооллыг нэгдсэн байдлаар хийх боломжтой.
\end{itemize}

\subsubsection{API загварын ерөнхий зарчим}

API дизайнд дараах зарчмыг баримталсан:

\begin{itemize}
	\item resource-oriented endpoint (жишээ нь cards, articles, dictionary, contests);
	\item CRUD урсгалын стандарчлал (create/read/update/delete);
	\item authentication/authorization-ийг role-той уялдуулах;
	\item frontend талд дахин ашиглахад хялбар, тогтвортой response бүтэц.
\end{itemize}

\subsection{Технологийн сонголт}

Технологийн сонголтыг дараах шалгуурын дагуу хийсэн:
\begin{itemize}
	\item өргөтгөх боломж;
	\item хурд болон гүйцэтгэл;
	\item олон платформ дэмжих чадвар;
	\item хөгжүүлэлтийн үр ашиг.
\end{itemize}

Backend-д Supabase (PostgreSQL) ашигласан бол front-end талд веб болон мобайл орчинд тохирох framework сонгож, API-д суурилсан холболтыг хэрэгжүүлсэн.

Технологийн сонголтын үндэслэлийг илүү нарийвчлан авч үзвэл:

\begin{itemize}
	\item \textbf{Supabase/PostgreSQL}: сургалтын өгөгдөл, хэрэглэгчийн контент, ранкингийн тооцоололд реляц загвар тохиромжтой;
	\item \textbf{Cloud storage}: зураг, аудио, медиа контентыг өгөгдлийн сангаас тусгаарлан хадгалах;
	\item \textbf{REST API}: веб ба мобайлд нийтлэг өгөгдлийн эх үүсвэр бий болгох;
	\item \textbf{Модульчлагдсан frontend}: шинэ функц нэмэх үед тусгаар хөгжүүлэх боломжтой байх.
\end{itemize}

\subsection{SRS алгоритмын аргачлал}

FutureHub-ийн флашкарт давталтын хэсэгт карт бүрийн давталтын интервал, хүндрэл, зөв/буруу хариултын түүхийг харгалзан дараагийн давталтыг тодорхойлох зарчмыг ашиглав. Ерөнхий ойлголтоор:

\[
I_{n} = I_{n-1} \times EF
\]

энд $I_n$ нь дараагийн интервал, $EF$ нь тухайн картын амархан/хэцүү байдлыг илэрхийлэх коэффициент болно. Хэрэглэгч зөв хариулж тогтвортой ахиж байгаа үед интервал уртсаж, алдаа ихтэй үед интервал богиносох байдлаар давталт адаптив болно.

Энэ логик нь:

\begin{itemize}
	\item өдөр бүр ижил тооны карт механикаар давтах ачааллыг бууруулж,
	\item картын хэцүү түвшинд тохируулан суралцах үр ашгийг нэмэгдүүлж,
	\item хэрэглэгчийн цагийн нөөцөд нийцсэн ухаалаг давталт үүсгэнэ.
\end{itemize}

\subsection{OCR интеграцийн аргачлал}

OCR урсгалыг дараах байдлаар арга зүйчилсэн:

\begin{enumerate}[leftmargin=2em]
	\item хэрэглэгч зураг оруулах;
	\item танигдсан текстийг урьдчилан засварлах интерфейсээр баталгаажуулах;
	\item баталгаажсан үгийг толь бичгийн хайлттай холбох;
	\item шинэ эсвэл сайжруулсан утгыг флашкарт руу хадгалах;
	\item үүсгэсэн картын давталтын эхний интервал автоматаар тохируулах.
\end{enumerate}

Энэ урсгал нь зураг дээрх нэр томьёог шууд суралцах нэгж болгож хөрвүүлэх боломж олгож, ялангуяа япон хэл дээрх материалтай ажиллах үед хэрэглэгчийн ачааллыг бууруулна.

\subsection{UI/UX загвар дизайн}

Хэрэглэгчдэд ээлтэй, ойлгомжтой интерфейс боловсруулах зорилгоор wireframe болон prototype загваруудыг боловсруулсан. Gamified review (Quizlet, Duolingo маягийн) механикийг ашиглан давталтыг илүү сонирхолтой болгох дизайн шийдлийг тусгасан.

UI/UX боловсруулахдаа дараах зарчмыг ашигласан:

\begin{itemize}
	\item \textbf{Task-first design}: хэрэглэгч нэг даалгаврыг хамгийн бага алхмаар гүйцэтгэх;
	\item \textbf{Consistency}: вэб ба мобайл дээр нэршил, үйлдлийн логик ижил байх;
	\item \textbf{Feedback visibility}: үйлдлийн үр дүн (хадгалсан, баталгаажсан, оноо өссөн гэх мэт)-г шууд харагдуулах;
	\item \textbf{Progress awareness}: хэрэглэгч өдөр тутмын ахиц, үлдсэн даалгавраа шууд харах.
\end{itemize}

Эдгээр зарчим нь суралцах орчинд хэрэглэгчийн когнитив ачааллыг бууруулж, тогтвортой ашиглалт бий болгох зорилготой.

\subsection{Хөгжүүлэлт}

Системийн хөгжүүлэлтийг модульчлагдсан байдлаар хийсэн. Үүнд:

\begin{itemize}
	\item Flashcard module;
	\item Dictionary module;
	\item Blog / Article module;
	\item Discussion module;
	\item Project log module;
	\item Contest module;
	\item Profile \& Contribution ranking module;
\end{itemize}

Веб болон мобайл хувилбарууд ижил backend API ашиглан өгөгдлийг синхрончлох байдлаар хэрэгжүүлсэн.

\subsubsection{Модуль тус бүрийн хөгжүүлэлтийн зарчим}

\begin{itemize}
	\item \textbf{Flashcard module}: картын lifecycle (үүсгэх, давтах, архивлах)-ийг төв болгож хөгжүүлсэн;
	\item \textbf{Dictionary module}: нэр томьёоны утга, орчуулга, жишээ, эх сурвалжийн бүтэц;
	\item \textbf{Blog/Article module}: markdown-т төвлөрсөн хөнгөн агуулгын урсгал;
	\item \textbf{Discussion module}: асуулт-хариулт ба саналын үнэлгээ;
	\item \textbf{Project log module}: хугацааны дарааллаар ажлын явц бичих боломж;
	\item \textbf{Contest/Course module}: оролцоо, хугацаа, дүнгийн үндсэн логик;
	\item \textbf{Ranking module}: олон эх сурвалжаас ирэх хувь нэмрийн оноог нэгтгэн тооцох.
\end{itemize}

Хөгжүүлэлтийн явцад модуль бүрийг тусгаар туршиж, дараа нь API түвшинд нэгтгэх хэлбэрээр алдааны нөлөөллийг багасгасан.

\subsection{Туршилт ба үнэлгээ}

Системийн туршилтыг дараах түвшинд хийсэн:
\begin{itemize}
	\item Unit test (тус бүрийн функц шалгах);
	\item Integration test (модуль хоорондын уялдаа);
	\item User acceptance test (хэрэглэгчийн туршилт).
\end{itemize}

Туршилтын үр дүнд илэрсэн алдаа, гүйцэтгэлийн асуудлуудыг засварлаж, UI болон алгоритмын сайжруулалт хийсэн.

\subsubsection{Туршилтын сценарийн бүтэц}

Туршилтын сценарийг дараах бүтэцтэй боловсруулсан:

\begin{enumerate}[leftmargin=2em]
	\item урьдчилсан нөхцөл (хэрэглэгчийн эрх, өгөгдлийн төлөв),
	\item туршилтын алхам,
	\item хүлээгдэж буй үр дүн,
	\item бодит үр дүн,
	\item зөрүү ба сайжруулалтын арга хэмжээ.
\end{enumerate}

Энэ бүтэц нь зөвхөн алдаа илрүүлэх бус, яагаад алдаа үүсэж байгааг логикоор тайлбарлахад тусалсан.

\subsubsection{Үнэлгээний шалгуур}

Прототипын түвшинд үнэлгээг дараах гол шалгуураар хийв:

\begin{itemize}
	\item хэрэглэгчийн гол урсгалууд тасралтгүй ажиллах эсэх;
	\item модуль хооронд өгөгдлийн дамжуулалт зөв ажиллах эсэх;
	\item SRS давталтын төлөв логик дарааллаар шинэчлэгдэх эсэх;
	\item OCR-оор үүсгэсэн контент толь бичиг болон карт руу зөв шилжих эсэх;
	\item ранкингийн тооцоолол хэрэглэгчийн үйлдэлтэй уялдах эсэх.
\end{itemize}

\section{Бүлгийн дүгнэлт}

Энэхүү бүлэгт FutureHub платформыг боловсруулахад ашигласан судалгааны арга зүйг шаардлага тодорхойлолтоос эхлэн архитектур, алгоритм, хөгжүүлэлт, туршилтын түвшин хүртэл үе шаттайгаар тодорхойлов. SDLC-ийн итератив хандлага нь хэрэглэгчийн бодит хэрэгцээнд ойр, өргөтгөх боломжтой, веб-мобайл уялдаа сайтай суурь прототип бүтээхэд тохиромжтой болохыг харуулж байна.

Дараагийн бүлэгт дээрх арга зүйгээр хэрэгжүүлсэн модуль тус бүрийн үр дүн, системийн ажиллах урсгал, туршилтын дүгнэлтийг дэлгэрүүлэн танилцуулна.