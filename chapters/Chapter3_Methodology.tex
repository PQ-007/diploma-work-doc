\chapter{Судалгааны арга зүй}
\label{ch:methodology}

\section{Ерөнхий аргачлал}
Энэхүү ажлын хүрээнд “FutureHub” системийг боловсруулахдаа дараах үе шаттай аргачлалыг баримталсан:
\begin{enumerate}
	\item шаардлага тодорхойлох ба асуудал шинжилгээ;
	\item системийн загварчлал (үүрэг-орчин, хэрэглээний урсгал, өгөгдлийн загвар);
	\item архитектурын шийдэл боловсруулах;
	\item прототип хэрэгжүүлэлт (веб + мобайл);
	\item туршилт, үнэлгээ, сайжруулалт.
\end{enumerate}

\section{Шаардлага тодорхойлох}
Шаардлагыг тодорхойлохдоо дипломын ажлын зорилго, зорилтууд болон хэрэглэгчдийн үндсэн асуудлууд (нэр томьёо тогтоолт, орчуулга/тайлбарын хомсдол, мэдээллийн тархай байдал, баримтжуулалтын дутмаг байдал)-д тулгуурлан функциональ шаардлагуудыг бүлэглэв. Үүнд:
\begin{itemize}
	\item \textbf{Мобайл (флашкарт)}: карт үүсгэх/засах/устгах, давталтын хуваарь, аудио давталт, OCR-аар карт үүсгэх,
	\item \textbf{Веб (контент)}: нийтлэл/блог, асуулт-хариулт, толь бичиг, төслийн тэмдэглэл,
	\item \textbf{Веб (орчин)}: тэмцээн, сургалт, ранкинг, админ/багшийн хяналт.
\end{itemize}

\section{Системийн загварчлал}
Системийн оролцогч талуудыг гурван үндсэн үүргээр авч үзэв: \textit{хэрэглэгч (оюутан)}, \textit{багш}, \textit{админ}. Зураг~\ref{fig:futurehub-context}-т веб ба мобайл хэсгийн гол модулиуд болон оролцогчдын харилцан үйлчлэлийг ерөнхий байдлаар үзүүлсэн.

\subsection{Хэрэглээний урсгалын товч тайлбар}
Диаграммын агуулгыг үйл ажиллагааны түвшинд дараах байдлаар нэгтгэнэ:
\begin{itemize}
	\item \textbf{Нийтлэл / Блог}: хэрэглэгч нийтлэл унших, бичих; багш чанарын хяналт хийх боломжтой.
	\item \textbf{Толь бичиг}: нэр томьёо нэмэх/засах/хайх; админ хянах.
	\item \textbf{Флашкарт}: үг хэллэгийг үүсгэх, давтах, ахиц хянах; OCR ашиглан зурагнаас карт үүсгэх урсгал.
	\item \textbf{Асуулт-хариулт}: хэрэглэгч асуулт асуух, хариулах; багш чиглүүлэх.
	\item \textbf{Төсөл боловсруулах}: төслийн явцын тэмдэглэл үүсгэх, баримтжуулах; админ/багш хянах.
	\item \textbf{Тэмцээн, сургалт, ранкинг}: оролцоо, оноолт, үнэлгээний механизмууд.
\end{itemize}

\section{Архитектурын шийдэл}
FutureHub нь хэрэглэгчийн төхөөрөмж дээр ажиллах \textit{мобайл аппликейшн}, хөтөч дээр ажиллах \textit{веб интерфэйс}, мөн өгөгдөл, бизнес логикийг удирдах \textit{сервер талын бүрэлдэхүүн} гэсэн гурван үндсэн хэсэгтэй гэж төлөвлөв.

\begin{itemize}
	\item \textbf{Клиент тал}: мобайл апп (флашкарт, OCR урсгал), веб платформ (нийтлэл, толь, Q\&A, project log гэх мэт).
	\item \textbf{Сервер тал}: хэрэглэгчийн өгөгдөл, контент, үнэлгээ/ранкинг, эрхийн удирдлагыг нэг API давхаргаар хангана.
	\item \textbf{Өгөгдлийн сан}: хэрэглэгч, контент, толь бичиг, флашкарт, үйл ажиллагааны бүртгэл, үнэлгээ зэрэг өгөгдлийг бүтэцтэй хадгална.
\end{itemize}

\section{Өгөгдлийн сангийн загварчлал}
Өгөгдлийн сангийн загварчлалд дараах үндсэн объектуудыг (entity) тодорхойлж, хоорондын хамаарлыг төлөвлөв: \textit{User}, \textit{Role}, \textit{Article/Post}, \textit{DictionaryTerm}, \textit{Flashcard}, \textit{ReviewLog}, \textit{ProjectLog}, \textit{Question/Answer}, \textit{Contest/Training}, \textit{ContributionScore}. Энэ нь контентын модуль бүр бие даан хөгжих боломжтой байх, мөн нийт ранкинг/үнэлгээний системд нэгтгэгдэх боломжийг хангана.

\section{Туршилт ба үнэлгээ}
Туршилтыг дараах хүрээнд авч үзэв:
\begin{itemize}
	\item \textbf{Функциональ туршилт}: модуль бүрийн үндсэн CRUD үйлдэл, хэрэглээний урсгалын алдааг шалгах;
	\item \textbf{Хэрэглэгчийн туршлага (UX)}: флашкарт давталтын урсгал, хайлт, контент унших/бичихийн ойлгомжтой байдал;
	\item \textbf{Интеграцийн туршилт}: веб ба мобайл хэрэглээнээс нэг өгөгдлийн эх үүсвэр (сервер/API, өгөгдлийн сан)-тэй харилцах уялдаа.
\end{itemize}

