\chapter{Удиртгал}
\label{ch:introduction}

\section{Судалгааны үндэслэл}
КУТ-ийн оюутнууд мэргэжлийн хичээл судлах, япон хэл дээрх сурах бичиг болон эх сурвалж ашиглах, багийн төсөл хэрэгжүүлэх, судалгааны ажил хийх явцдаа дараах нийтлэг хүндрэлүүдтэй тулгардаг. Үүнд: мэргэжлийн нэр томьёоны ойлголт хангалтгүй байх, чанартай орчуулгатай агуулга хомс байх, мэдээллийн эх сурвалж тархай бутархай байх, мөн хийсэн ажлаа системтэй баримтжуулж ур чадвараа нотлох нэгдсэн орчин дутмаг байх зэрэг асуудал багтана.

Дээрх асуудлууд нь суралцах үйл явцыг тасалдуулах, бие даан ахих хурдыг бууруулах, хамтын суралцах боломжийг хязгаарлах сөрөг нөлөөтэй. Иймээс унших, суралцах материал, нэр томьёоны сан, хамтын ажиллагаа, төсөл суурьтай баримтжуулалт болон урамшууллын механизмыг нэгтгэсэн дижитал орчин хэрэгцээтэй байна.

\section{Асуудлын тодорхойлолт} Одоогийн нөхцөлд оюутнууд:

\begin{itemize}[leftmargin=2em, itemsep=0.2em]

\item \textbf{Нэр томьёо тогтоолтын хүндрэл}\\
Япон хэл дээрх мэргэжлийн нэр томьёог тогтоох, тогтмол давтах, давталтын хуваарь баримтлахад хүндрэлтэй.

\item \textbf{Мэдээллийн тархмал байдал}\\
Орчуулга, тайлбар, жишээ, хэрэглээ зэрэг мэдээлэл нэг дор төвлөрөөгүй тул хайлт хийхэд ихээхэн хугацаа зарцуулдаг.

\item \textbf{Баримтжуулалтын стандарт дутмаг}\\
Төсөл, лабораторийн ажил, судалгааны ажлын явцыг нэгдсэн стандартын дагуу баримтжуулах орчин хангалтгүй.

\item \textbf{Хамтын орчны задрал}\\
Асуулт-хариулт, нийтлэл, блог, тэмцээн зэрэг хамтын үйл ажиллагаа тархмал, нэгдсэн системгүй байна.
\end{itemize}

Эдгээрийг шийдвэрлэх зорилгоор “FutureHub” нэртэй веб болон мобайл цогц платформын загварчлал, хөгжүүлэлтийн арга зүйг судалж, прототип шийдэл боловсруулах шаардлага үүссэн.

\section{Судалгааны зорилго ба зорилтууд}

\subsection{Зорилго}
Энэхүү дипломын ажлын зорилго нь КУТ-ийн оюутнуудад зориулсан “FutureHub” нэртэй суралцах үйл явцыг дэмжих веб ба мобайл цогц платформын шийдэл боловсруулахад оршино.

\subsection{Зорилтууд}
Судалгааны зорилгыг хангахын тулд дараах зорилтуудыг дэвшүүлэв.
\begin{enumerate}[label=\textbf{\arabic*.},
                  leftmargin=2em, itemsep=0.35em, topsep=0.35em]
	\item Системийн хэрэглэгчдийн хэрэгцээ, асуудлыг тодорхойлж функциональ болон функциональ бус шаардлагыг боловсруулах.
	\item Флашкарт дээр суурилсан үг, нэр томьёо тогтоолтын модулийг \emph{Spaced Repetition System (SRS)} зарчимд тулгуурлан төлөвлөх.
	\item \emph{OCR} ашиглан зургаас текст таних, толь бичиг/нэр томьёоны санг дэмжих шийдлийг тодорхойлох.
	\item Нийтлэл, блог, хэлэлцүүлэг, төслийн тэмдэглэл, тэмцээн, сургалт, ранкинг зэрэг веб модулиудын интеграцчилсан загварыг боловсруулах.
	\item Архитектур, өгөгдлийн сангийн зохион байгуулалт, прототип хөгжүүлэлт, туршилт-үнэлгээний ерөнхий аргачлалыг хэрэгжүүлэх.
\end{enumerate}

\section{Судалгааны хамрах хүрээ}

Энэхүү ажил нь Шинэ Монгол Технологийн коллежийн КУТ-ын хүрээнд боловсролын зориулалттай веб ба мобайл платформын \textit{суурь шийдэл} боловсруулахад төвлөрсөн. Судалгааны үндсэн зорилтот хэрэглэгч нь КУТ-ийн оюутнууд бөгөөд багш, админ хэрэглэгчийн төрлийг системийн бүтэц дотор тусгасан.

Судалгааны хүрээнд флашкарт давталтын алгоритм (Spaced Repetition System), мэргэжлийн толь бичиг, нийтлэл/блогийн систем, сургалтын модуль, тэмцээн уралдаан, хэлэлцүүлгийн хэсэг, хэрэглэгчдийг үнэлэх систем (Contribution / Competition Ranking), профайл портфолио зэрэг үндсэн функцуудын архитектур, өгөгдлийн зохион байгуулалт болон харилцан хамаарлыг хамруулсан.

Систем нь веб болон мобайл гэсэн хоёр client хэсгээс бүрдэх ба backend API-аар PostgreSQL өгөгдлийн сантай холбогдон ажиллах туршилтын загвар (prototype) байдлаар хөгжүүлэгдсэн.

Энэхүү судалгаа нь системийн бүрэн нэвтрүүлэлт, өргөтгөсөн аюулгүй байдлын аудит, олон тооны (1000-аас дээш) хэрэглэгчийн ашиглалтанд суурилсан статистик үнэлгээ зэргийг хамраагүй болно.

\section{Ажлын ач холбогдол ба шинэлэг тал}
Санал болгож буй шийдэл нь нэг платформ дээр мэргэжлийн нэр томьёо бүхий толь бичиг, OCR, мэдлэг хуваалцах нийтлэл/блог, асуулт хариулт, төслийн явцын тэмдэглэл, тэмцээн/сургалт зэрэг боломжийг нэгтгэснээр суралцах үйл явцыг илүү системтэй болгож, оюутан бие даан суралцаж, хамтран хөгжиж амжилт гаргахад нь туслана гэж үзэж байна.

Гол онцлох шинэлэг тал нь компьютерийн ухааны хөтөлбөрийн roadmap-ийг харах боломжтой байдал бөгөөд ингэснээр оюутнууд цаашдын мэргэших чиглэлээ сонгон, өөрийн суралцах замаа төлөвлөх,  шаардлагатай мэдлэг, ур чадварын хоорондын хамаарлыг ойлгох, өөрийн ахиц дэвшлийг бодитоор харах боломжтой.

\section{Судалгааны бүтэц}

Энэхүү cудалгаа нь таван бүлгээс бүрдэнэ. Үүнд:

\begin{itemize}
    \item \textbf{Нэгдүгээр бүлэг} – Асуудлын үндэслэл, зорилго, хамрах хүрээг тодорхойлно.
    
    \item \textbf{Хоёрдугаар бүлэг} – Холбогдох онол, судалгааны ажлуудын тоймыг харуулна.
    
    \item \textbf{Гуравдугаар бүлэг} – Систем боловсруулах арга зүй, загварчлал, архитектурын шийдлийг тайлбарлана.
    
    \item \textbf{Дөрөвдүгээр бүлэг} – Хөгжүүлэлтийн үр дүн, туршилт, үнэлгээний дүгнэлтийг нэгтгэнэ.
    
    \item \textbf{Тавдугаар бүлэг} – Нийт дүгнэлт болон цаашдын хөгжүүлэлтийн чиглэлийг тодорхойлно.
\end{itemize}
