\chapter{Удиртгал}
\label{ch:introduction}

\section{Судалгааны үндэслэл}
КУТ-ийн оюутнууд мэргэжлийн хичээл судлах, япон хэл дээрх сурах бичиг болон эх сурвалж ашиглах, багийн төсөл хэрэгжүүлэх, судалгааны ажил хийх явцдаа дараах нийтлэг хүндрэлүүдтэй тулгардаг. Үүнд: мэргэжлийн нэр томьёоны ойлголт хангалтгүй байх, чанартай орчуулгатай агуулга хомс байх, мэдээллийн эх сурвалж тархай бутархай байх, мөн хийсэн ажлаа системтэй баримтжуулж ур чадвараа нотлох нэгдсэн орчин дутмаг байх зэрэг асуудал багтана.

Дээрх асуудлууд нь суралцах үйл явцыг тасалдуулах, бие даан ахих хурдыг бууруулах, хамтын суралцах боломжийг хязгаарлах сөрөг нөлөөтэй. Иймээс суралцах материал, нэр томьёоны сан, хамтын ажиллагаа, төсөл-суурьтай баримтжуулалт болон урамшууллын механизмыг нэгтгэсэн дижитал орчин хэрэгцээтэй байна.

\section{Асуудлын тодорхойлолт}
Одоогийн нөхцөлд оюутнууд:
\begin{itemize}
	\item Япон хэл дээрх мэргэжлийн нэр томьёог тогтоох, тогтмол давтах, давталтын хуваарь баримтлахад хүндрэлтэй;
	\item Орчуулга, тайлбар, жишээ, хэрэглээ зэрэг мэдээлэл нэг дор төвлөрөөгүй тул хайлт хийхэд хугацаа их зарцуулдаг;
	\item Төсөл, лабораторийн ажил, судалгааны ажлын явцыг нэг стандартаар баримтжуулах орчин дутмаг;
	\item Асуулт-хариулт, нийтлэл, блог, тэмцээн зэрэг хамтын орчны үйл ажиллагаа тархмал байдаг.
\end{itemize}

Эдгээрийг шийдвэрлэх зорилгоор “FutureHub” нэртэй веб болон мобайл цогц платформын загварчлал, хөгжүүлэлтийн арга зүйг судалж, прототип шийдэл боловсруулах шаардлага үүссэн.

\section{Судалгааны зорилго ба зорилтууд}
\subsection{Зорилго}
Энэхүү дипломын ажлын зорилго нь КУТ-ийн оюутнуудад зориулсан “FutureHub” нэртэй суралцах үйл явцыг дэмжих веб ба мобайл цогц платформын шийдэл боловсруулахад оршино.

\subsection{Зорилтууд}
Судалгааны зорилгыг хангахын тулд дараах зорилтуудыг дэвшүүлэв:
\begin{enumerate}
	\item Системийн хэрэглэгчдийн хэрэгцээ, асуудлыг тодорхойлж функциональ болон функциональ бус шаардлагыг боловсруулах;
	\item Флашкарт дээр суурилсан үг, нэр томьёо тогтоолтын модулийг зайтай давталтын (SRS) зарчимд тулгуурлан төлөвлөх;
	\item OCR ашиглан зургаас текст таних, толь бичиг/нэр томьёоны санг дэмжих шийдлийг тодорхойлох;
	\item Нийтлэл, блог, хэлэлцүүлэг, төслийн тэмдэглэл, тэмцээн, сургалт, ранкинг зэрэг веб модулиудын интеграцчилсан загварыг боловсруулах;
	\item Архитектур, өгөгдлийн сангийн зохион байгуулалт, прототип хөгжүүлэлт, туршилт үнэлгээний ерөнхий аргачлалыг хэрэгжүүлэх.
\end{enumerate}

\section{Судалгааны хамрах хүрээ}
Энэхүү ажил нь боловсролын зориулалттай веб ба мобайл платформын \textit{суурь шийдэл} боловсруулахад төвлөрсөн. Системийн бүрэлдэхүүн хэсгүүдийг хэрэглэгчийн үүрэг (оюутан/хэрэглэгч, багш, админ) болон үндсэн хэрэглээний урсгал (унших, бичих, үүсгэх, хянах, оролцох) дээр тулгуурлан тодорхойлов.

\begin{figure}[H]
\centering
\IfFileExists{pictures/futurehub-context.png}{
	\includegraphics[width=0.95\linewidth]{pictures/futurehub-context.png}
}{
	\fbox{\parbox{0.92\linewidth}{\centering
	\vspace{3mm}
	\textbf{Зураг оруулаагүй байна.}\\
	Дараах файлыг \texttt{pictures/} хавтсанд байршуулна уу: \texttt{futurehub-context.png}\\
	(Веб сайт + Гар утасны апп-ийн ерөнхий бүтэц бүхий зураг)
	\vspace{3mm}
	}}
}
\caption{FutureHub системийн ерөнхий бүтэц (веб ба мобайл бүрэлдэхүүн)}
\label{fig:futurehub-context}
\end{figure}

\section{Ажлын ач холбогдол ба шинэлэг тал}
Санал болгож буй шийдэл нь нэг платформ дээр нэр томьёо тогтоолт (флашкарт/SRS), толь бичиг, OCR, мэдлэг хуваалцах нийтлэл/блог, асуулт-хариулт, төслийн явцын тэмдэглэл (project log), тэмцээн/сургалт болон хувь нэмрийн үнэлгээ (ранкинг) зэрэг боломжийг нэгтгэснээр суралцах үйл явцыг илүү системтэй болгож, хамтын суралцах орчинг дэмжинэ.

\section{Тайлангийн бүтэц}
Энэхүү тайлан нь таван бүлгээс бүрдэнэ. Нэгдүгээр бүлэгт асуудлын үндэслэл, зорилго, хамрах хүрээг тодорхойлно. Хоёрдугаар бүлэгт холбогдох онол, судалгааны ажлуудын тоймыг харуулна. Гуравдугаар бүлэгт систем боловсруулах арга зүй, загварчлал, архитектурын шийдлийг тайлбарлана. Дөрөвдүгээр бүлэгт хэрэгжүүлэлтийн үр дүн, туршилт үнэлгээний дүгнэлтийг нэгтгэнэ. Тавдугаар бүлэгт нийт дүгнэлт болон цаашдын хөгжүүлэлтийн чиглэлийг санал болгоно.
