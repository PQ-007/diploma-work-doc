\chapter{Удиртгал}
\label{ch:introduction}

\section{Судалгааны үндэслэл}
КУТ-ийн оюутнууд мэргэжлийн хичээл судлах, япон хэл дээрх сурах бичиг болон эх сурвалж ашиглах, багийн төсөл хэрэгжүүлэх, судалгааны ажил хийх явцдаа олон давхар хүндрэлтэй тулгардаг. Эдгээр хүндрэл нь зөвхөн нэг хичээл эсвэл нэг семестрийн хүрээний асуудал бус, харин суралцах дадал, мэдлэгийн хуримтлал, хамтын бүтээмж, карьерын бэлтгэлтэй шууд холбоотой урт хугацааны асуудал юм.

Практик ажиглалтаар дараах нөхцөлүүд түгээмэл байна. Нэгдүгээрт, мэргэжлийн нэр томьёо болон япон-англи-монгол орчуулгын нэгдсэн сангүйгээс үүдэн нэг ойлголтыг олон эх сурвалжаас хайх шаардлага тогтмол гардаг. Хоёрдугаарт, нэр томьёог цээжлэх үйл явц нь системгүй, санамсаргүй давталттай тул богино хугацаанд сурсан мэдээлэл хурдан мартагдах хандлагатай. Гуравдугаарт, оюутнуудын хийж буй төсөл, лаборатори, бие даалт нь ур чадварын нотолгоо болж чадахуйц бүтцээр хадгалагддаггүй тул дараагийн шатны суралцах болон ажилд орох бэлтгэлд мэдээллийн хоосон зай үүсдэг.

Эдгээр нөхцөл байдал нь суралцах үйл явцыг тасалдуулах, бие даан ахих хурдыг бууруулах, хамтран суралцах боломжийг хязгаарлах сөрөг нөлөөтэй. Иймээс унших, суралцах материал, нэр томьёоны сан, хамтын ажиллагаа, төсөл суурьтай баримтжуулалт, урамшууллын механизмыг нэгтгэсэн дижитал орчин боловсруулах хэрэгцээ бодитоор үүсэж байна.

\section{Асуудлын тодорхойлолт}

Одоогийн нөхцөлд оюутнууд дараах үндсэн асуудлуудтай тулгарч байна.

\begin{itemize}[leftmargin=2em, itemsep=0.3em]

\item \textbf{Нэр томьёо тогтоолтын хүндрэл}\\
Япон хэл дээрх мэргэжлийн нэр томьёог тогтоох, тогтмол давтах, давталтын хугацааг өөрийн түвшинд тааруулахад хүндрэлтэй. Ихэнх тохиолдолд давталтын төлөвлөгөө эсвэл тэмдэглэлд хуваагдан үлддэг тул урт хугацааны тогтоолт буурдаг.

\item \textbf{Мэдээллийн тархмал байдал}\\
Орчуулга, тайлбар, жишээ, хэрэглээ, кодын тайлбар зэрэг материал нь олон платформд тархсан байдаг. Үүний улмаас нэг ойлголт эзэмшихэд шаардагдах хугацаа уртасч, суралцах урсгал тасалдах эрсдэл нэмэгддэг.

\item \textbf{Баримтжуулалтын стандарт дутмаг}\\
Төсөл, лаборатори, судалгааны ажлын явцыг нэгдсэн загвараар баримтжуулах орчин хязгаарлагдмал. Үүний улмаас хийсэн ажлын үнэ цэн, гаргаж буй ахицын логик дараалал, гүйцэтгэсэн үүргийн нотолгоо сул харагддаг.

\item \textbf{Хамтын орчны задрал}\\
Асуулт-хариулт, нийтлэл, туршлага хуваалцах, тэмцээн, сургалтын мэдээлэл нь нэгдсэн биш тул хамтран суралцах нөлөө буурч, нэг оюутны бүтээсэн мэдлэг нөгөө оюутанд хурдан хүрэхгүй байна.

\item \textbf{Урамшууллын тогтолцооны сул байдал}\\
Хэрэглэгчийн идэвх ба хийж буй ажил бусдад хэрэгтэй байдлыг нь тооцож харуулах нэгдсэн ранкинг, амжилтын үзүүлэлтүүд хангалтгүйгээс тогтвортой оролцоо, хийж бүтээх сэдэл буурдаг.
\end{itemize}

\subsection{Асуудлын шалтгаан-үр дагаврын шинжилгээ}

Дээрх асуудлуудыг шалтгаан-үр дагаврын үүднээс авч үзвэл: (i) контентын нэгдсэн бүтэцгүй байдал, (ii) давталт ба ахицын өгөгдөлд суурилсан системгүй байдал, (iii) оролцоог үнэлэх механизмын сул байдал гэсэн гурван суурь шалтгаан илэрч байна. Эдгээр шалтгааны нөлөөгөөр оюутны суралцах хугацааны үр ашиг буурч, бие даан суралцах дадал тогтворжихгүй, багийн хамтын бүтээмж буурах нөхцөл үүсдэг.

Иймээс “FutureHub” платформын үндсэн зорилт нь дээрх гурван суурь шалтгааныг нэг архитектур дотор шийдвэрлэхэд чиглэнэ. Өөрөөр хэлбэл, суралцах контентын удирдлага, ухаалаг давталт, хамтын оролцооны үнэлгээг интеграцчилсан систем болгон нэгтгэх шаардлага гарч байна.

\begin{figure}[htbp]
\centering
\includegraphics[width=0.95\textwidth]{pictures/Системийн шаардлага ба ерөнхий төсөөлөл.png}
\caption{FutureHub системийн шаардлага ба ерөнхий төсөөлөл}
\label{fig:requirements-overview}
\end{figure}

Зураг~\ref{fig:requirements-overview}-т системийн шаардлагын ерөнхий хүрээ, хэрэглэгчийн төрөл, гол үйл ажиллагаанууд болон үнэ цэнийн санал (value proposition)-ыг нэгтгэн үзүүлсэн. Энэхүү дүрслэл нь дараагийн бүлгүүдэд тайлбарлах архитектур, модулийн шийдлийн суурь ойлголт болно.

\section{Судалгааны зорилго ба зорилтууд}

\subsection{Зорилго}
Энэхүү дипломын ажлын зорилго нь КУТ-ийн оюутнуудад зориулсан “FutureHub” нэртэй суралцах үйл явцыг дэмжих веб ба мобайл цогц платформын шийдэл боловсруулж, суурь прототип түвшинд хөгжүүлэн үнэлэхэд оршино.

\subsection{Судалгааны асуултууд}

Судалгааны зорилгыг тодорхой болгохын тулд дараах үндсэн асуултуудыг дэвшүүлэв.

\begin{enumerate}[label=\textbf{RQ\arabic*.}, leftmargin=2.3em, itemsep=0.35em]
    \item Нэр томьёо тогтоолтыг SRS-д суурилсан давталтын механизмаар сайжруулах боломж хэр байна вэ?
    \item OCR + толь бичгийн интеграцчилал нь контент үүсгэх хугацаа, хэрэглэгчийн ачаалалд ямар нөлөө үзүүлэх вэ?
    \item Нийтлэл, хэлэлцүүлэг, төслийн тэмдэглэл, тэмцээн, ранкингийг нэг системд нэгтгэх нь хамтын суралцах орчинд ямар үнэ цэн бий болгох вэ?
    \item Веб ба мобайл орчны өгөгдлийн синхрончлолыг хөнгөн архитектураар найдвартай хөгжүүлэх боломжтой юу?
\end{enumerate}

\subsection{Зорилтууд}

Судалгааны зорилгыг хангахын тулд дараах зорилтуудыг үе шаттай хэрэгжүүлэв.

\begin{enumerate}[label=\textbf{\arabic*.}, leftmargin=2em, itemsep=0.35em, topsep=0.35em]
    \item Системийн хэрэглэгчдийн хэрэгцээ, асуудлыг тодорхойлж функциональ болон функциональ бус шаардлагыг боловсруулах.
    \item Флашкарт дээр суурилсан үг, нэр томьёо тогтоолтын модулийг \emph{Spaced Repetition System (SRS)} зарчимд тулгуурлан төлөвлөх.
    \item \emph{OCR} ашиглан зургаас текст таних, толь бичиг/нэр томьёоны санг дэмжих шийдлийг тодорхойлох.
    \item Нийтлэл, блог, хэлэлцүүлэг, төслийн тэмдэглэл, тэмцээн, сургалт, ранкинг зэрэг веб модулиудын интеграцчилсан загварыг боловсруулах.
    \item Архитектур, өгөгдлийн сангийн зохион байгуулалт, прототип хөгжүүлэлт, туршилт-үнэлгээний аргачлалыг хэрэгжүүлэх.
    \item Үр дүнг чанарын түвшинд үнэлж, цаашдын хөгжүүлэлтийн чиглэлийг тодорхойлох.
\end{enumerate}

\subsection{Зорилт–үр дүнгийн уялдаа}

Зорилтуудыг биелүүлэх шалгуурыг дараах байдлаар тодорхойлов: (i) модуль бүрийн үндсэн урсгал ажиллах, (ii) веб-мобайл өгөгдлийн синхрончлол тасралтгүй байх, (iii) хэрэглэгчийн үйлдлүүдээс ранкингийн оноо тооцох логик хэрэгжсэн байх, (iv) OCR-оор үүсгэсэн картын урсгал практик хэрэглээнд ажиллах. Энэхүү шалгуур нь ажлын үр дүнгийн бүлэгт чанарын болон хэрэглээний түвшний үнэлгээ хийх үндэс болно.

\section{Судалгааны хамрах хүрээ}

Энэхүү ажил нь Шинэ Монгол Технологийн коллежийн КУТ-ын хүрээнд боловсролын зориулалттай веб ба мобайл платформын \textit{суурь шийдэл} боловсруулахад төвлөрсөн. Судалгааны үндсэн зорилтот хэрэглэгч нь КУТ-ийн оюутнууд юм.

Судалгааны хүрээнд дараах хэсгүүдийг хамруулан хөгжүүлсэн.

\begin{itemize}[leftmargin=2em]
    \item \textbf{Суралцах модуль:} флашкарт, зайтай давталт, толь бичиг, OCR;
    \item \textbf{Хамтын мэдлэгийн модуль:} нийтлэл, асуулт-хариулт, хэлэлцүүлэг;
    \item \textbf{Баримтжуулалтын модуль:} төслийн тэмдэглэл, профайл, портфолио;
    \item \textbf{Урамшууллын модуль:} тэмцээн, сургалт, contribution/competition ranking.
\end{itemize}

Систем нь веб болон мобайл гэсэн хоёр client хэсгээс бүрдэх ба backend API-аар PostgreSQL өгөгдлийн сантай холбогдон ажиллах туршилтын загвар (prototype) байдлаар хөгжүүлэгдсэн.

\subsection{Судалгааны хязгаарлалт}

Энэхүү судалгаа нь дараах хүрээг хамраагүй болно.

\begin{itemize}[leftmargin=2em]
    \item Системийн бүрэн нэвтрүүлэлт, үйлдвэрлэлийн түвшний (production-grade) DevOps автоматжуулалт;
    \item Өргөтгөсөн аюулгүй байдлын аудит (penetration test, red team exercise);
    \item Олон тооны (1000-аас дээш) хэрэглэгчийн статистик суурьтай урт хугацааны үнэлгээ;
    \item Сургалтын байгууллагын бүх дотоод системтэй бүрэн интеграцчилал.
\end{itemize}

Гэсэн хэдий ч эдгээр хязгаарлалтыг тодорхой тусгаснаар ажлын хүрээ бодитой, хэрэгжихүйц, хамгаалалтын шаардлагад нийцсэн хэвээр байна.

\section{Ажлын ач холбогдол ба шинэлэг тал}

Санал болгож буй шийдэл нь нэг платформ дээр мэргэжлийн нэр томьёо бүхий толь бичиг, OCR, мэдлэг хуваалцах нийтлэл/блог, асуулт-хариулт, төслийн явцын тэмдэглэл, тэмцээн/сургалт зэрэг боломжийг нэгтгэснээр суралцах үйл явцыг илүү системтэй болгож, оюутан бие даан суралцаж, хамтран хөгжих нөхцөлийг бүрдүүлнэ.

Энэхүү ажлын ач холбогдлыг дараах түвшинд тодорхойлж болно.

\begin{itemize}[leftmargin=2em]
    \item \textbf{Боловсролын үнэ цэн:} оюутны өдөр тутмын суралцах дадлыг тогтвортой болгох;
    \item \textbf{Хэрэглээний үнэ цэн:} контент үүсгэх, хайх, дахин ашиглах хугацааг бууруулах;
    \item \textbf{Хамтын үнэ цэн:} мэдлэгийг хувь хүнээс баг, хамт олон руу шилжүүлэх хурдыг нэмэгдүүлэх;
    \item \textbf{Хувь хүний үнэ цэн:} профайл, портфолиогоор дамжуулан хөгжин сайжирч байгаагаа цаг хугацааны хувьд илүү тод харуулах.
\end{itemize}

Гол онцлох шинэлэг тал нь компьютерийн ухааны хөтөлбөрийн roadmap-той уялдсан контент экосистем, мөн веб-мобайл хосолсон хэрэглээнд тохирсон SRS + OCR + Contribution Ranking-ийн интеграцчилсан загвар юм. Энэ нь тус тусдаа платформуудад байдаг боломжуудыг нэг хэрэглэгчийн урсгалд нэгтгэж өгснөөр онцлогтой.

\section{Судалгааны бүтэц}

Энэхүү судалгаа нь таван бүлгээс бүрдэнэ. Үүнд:

\begin{itemize}
    \item \textbf{Нэгдүгээр бүлэг} – Судалгааны үндэслэл, асуудлын тодорхойлолт, зорилго, хамрах хүрээ, шинэлэг талыг тайлбарлана.
    
    \item \textbf{Хоёрдугаар бүлэг} – Холбогдох онол, төстэй системүүдийн харьцуулсан шинжилгээ, судалгааны хоосон зай (research gap)-г тодорхойлно.
    
    \item \textbf{Гуравдугаар бүлэг} – Систем боловсруулах арга зүй, архитектур, загварчлал, технологийн сонголт, туршилтын аргачлалыг дэлгэрүүлнэ.
    
    \item \textbf{Дөрөвдүгээр бүлэг} – Прототип хөгжүүлэлтийн үр дүн, модулийн түвшний тайлбар, туршилтын дүн шинжилгээг танилцуулна.
    
    \item \textbf{Тавдугаар бүлэг} – Нийт дүгнэлт, зорилтын биелэлт, хязгаарлалт, цаашдын хөгжүүлэлтийн боломжийг тодорхойлно.
\end{itemize}

